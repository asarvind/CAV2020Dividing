\documentclass[runningheads]{llncs}

\newcommand\numberthis{\addtocounter{equation}{1}\tag{\theequation}} % numbering equations

\usepackage{cite}
\usepackage[utf8]{inputenc}
\usepackage{graphicx}
\usepackage{amsmath}
\usepackage{amsfonts}
\usepackage{color}
\usepackage{amssymb}
\usepackage{mathtools}
\usepackage{float}
\usepackage{multirow}
\usepackage{caption}
\usepackage{algorithm2e}


\RestyleAlgo{boxruled}




%% Sets
\newcommand{\cnums}{\mathbb{C}} % complex numbers
\newcommand{\reals}{\mathbb{R}}
\newcommand{\rationals}{\mathbb{Q}}
\newcommand{\integers}{\mathbb{Z}} % integers
\newcommand{\double}{\mathbb{D}} % double precision floating point numbers
\newcommand{\intervals}{\mathbb{I}} % set of intervals with bounds in double and ${-\infty,\infty}$
\newcommand{\set}[1]{\left\{#1\right\}} % set
\newcommand{\st}[1]{\left|~#1\right.} % such that
\newcommand{\fe}[1]{ \mathbb{F}_{#1} } % mathematical expressions
\newcommand{\vars}{V} % variables
\newcommand{\ball}[2]{\mathbb{B}\rb{#1,#2}} % open box with a radius
\newcommand{\interval}[2]{\sqb{#1,#2}} % interval
\newcommand{\dom}[1]{\operatorname{Dom}\rb{#1}} % domain of a function expression

%% Braces
\newcommand{\rb}[1]{\left(#1\right)}
\newcommand{\sqb}[1]{\left[#1\right]}

%% Linear algebra
\newcommand{\mymatrix}[1]{\begin{bmatrix}#1\end{bmatrix}} % matrix
\newcommand{\identity}[1]{\mathcal{I}_{#1}} % identity matrix
\newcommand{\rows}[1]{\operatorname{rows}\left(#1\right)} % number of rows in a matrix
\newcommand{\cols}[1]{\operatorname{cols}\left(#1\right)} % number of columns in a matrix
\newcommand{\diag}[1]{\mathcal{D}\rb{#1}} % diagonal matrix

%% Operators
\newcommand{\abs}[1]{\left|#1\right|} % absolute value
\newcommand{\minsum}{\oplus}
\newcommand{\norm}[1]{\left\|#1\right\|}
\newcommand{\meet}[2]{#1\bigwedge #2} % Meet
\newcommand{\join}[2]{#1\bigvee #2} % Join
\newcommand{\conv}[1]{\operatorname{{Conv}}\rb{#1}}
\newcommand{\real}[1]{\operatorname{Re}\rb{#1}}  % real part
\newcommand{\imag}[1]{\operatorname{Im}\rb{#1}}  % imaginary part
\DeclareMathOperator*{\argmin}{arg\,min} % arg min
\newcommand{\card}[1]{\#{#1}}

%% notes
\newcommand{\todo}[1]{todo: {\color{red}#1}}

%----------------------------------------------------------------------
%% paper specific
%----------------------------------------------------------------------
%% sets
\newcommand{\init}{\Psi}  % initial set
\newcommand{\inpset}{{U}} % input set

%% symbols
\newcommand{\proj}[1]{\lambda_{#1}} % function symbol for projection on a dimension
\newcommand{\trfn}[1]{\mathbf{#1}} % trajectory function
\newcommand{\gen}{G} % generator
\newcommand{\cen}{c} % center
\newcommand{\bounds}{d} % bounds
\newcommand{\eig}{\Lambda}  % eigenvectors at origin
\newcommand{\zt}{Z} % zonotope tuple

%% functions
\newcommand{\grad}[1]{\nabla{#1}} % gradient
\newcommand{\hess}[1]{\operatorname{H}{#1}} % hessian
\newcommand{\iv}[2]{\mu\rb{#1}\rb{#2}} % interval bounds on a function expression
\newcommand{\ivmat}[1]{\mu\rb{#1}} % interval matrix converted from double
\newcommand{\tr}[2]{\mathbf{#1}\rb{#2}} % trajectory value
\newcommand{\reach}[3]{\mathbf{#1}_{#2}\rb{#3}} % reachable set
\newcommand{\taylor}[3]{\sigma\rb{#1,#2,#3}} % taylor error
\newcommand{\mi}[1]{\alpha\rb{#1}} % center of an interval matrix
\newcommand{\rad}[1]{\beta\rb{#1}} % error of an interval matrix
\newcommand{\stmat}[3]{\mathcal{A}_{#1}^{#2,#3}} % state action matrix
\newcommand{\inpmat}[3]{\mathcal{B}_{#1}^{#2,#3}} % input action matrix
\newcommand{\err}[3]{E_{#1,#2,#3}} % additive remainder term in linearization
\newcommand{\dismat}[3]{{#1}^{#2}_{#3}} % discrete time action matrix
\newcommand{\iz}[3]{\left\llbracket\rb{#1,#2,#3}\right\rrbracket} % interval zonotope
\newcommand{\lin}[3]{\mathcal{T}\rb{#1,#2,#3}} % interval zonotope
\newcommand{\ivflow}[5]{\mathcal{S}_{#1,#2}^{#3,#4}\rb{#5}} % flow approximation inside an interval
\newcommand{\flow}[5]{\mathcal{R}_{#1,#2}^{#3,#4}\rb{#5}} % flow approximation at a time point
\newcommand{\utaylor}[4]{\widehat{\sigma}\rb{#1,#2,#3,#4}} % upper bound on taylor error
\newcommand{\dopt}[1]{O\rb{#1}} % optimum division vector
\newcommand{\iouopt}[2]{P^{#1}\rb{#2}} % optimum iou of interval zonotopes
\DeclareMathOperator*{\refine}{\pi} % refine
\newcommand{\ztset}[1]{\left\llbracket{#1}\right\rrbracket}
\newcommand{\zerozon}[2]{\zt_{#1\times #2}\rb{0}}


\title{Intersection of Unions Flowpipe for Minimizing Multi-Directional Linearization Error}
\titlerunning{Intersection of Union Flowpipe for Minimizing Multi-Directional}
\author{}
\institute{}
%
\begin{document}    
\maketitle
%
\begin{abstract}
We propose a new efficient method for computing over approximation of
set of trajectories, called flowpipe, for a system described by
nonlinear differential equations.  Since reachable sets of linear
systems can be computed efficiently, piecewise linearization of
nonlinear ODEs is a commonly used while computing flowpipes.  But the
number of pieces required in piecewise linearization becomes
exponential in the dimension for minimizing linearization error below
any threshold value, which can be computationally intractable in high
dimensions.  Alternatively, we can fix the number of divisions, in
which case we need to optimize the way reach set is divided to
minimize the linearization error.  Since the linearization error is
multi-dimensional, this minimization is a multi-objective optimization
problem which does not have a single best solution.  However, an
previous method in literature proposed to minimize a single
dimensional measure while dividing the reachable sets, which does not
necessarily reduce the linearization error along all coordinates.  In
this paper, we propose a better method where we use multiple best
optimal ways of division, instead of a single way of division, for
minimizing linearization error along multiple directions.
Subsequently, we represent the resulting reachable sets as
intersection of unions (IoU), where each intersecting union
corresponds to the optimized division for each direction among
multiple directions.  We propose an algorithm to propogate the
resulting IoU of sets for flowpipe computation.  This way, we ensure
that the linearization error along is reduced significantly along
every direction is a set of multiple directions.  We evaluate our
algorithm on an illustrative example and two high dimensional real
world examples.  We demonstrate high increase in accuracy compared to
the earlier method of division and also other state-of-the-art methods
for flowpipe computation of nonlinear systems.
\end{abstract}

\section{Mathematical preliminaries}
\subsection{Notation}
The set of real numbers is $\reals$, integers is $\integers$, and
complex numbers is $\cnums$.  The subset of real numbers that can be
represented in double precision floating point format is $\double$.
For $\Psi\subseteq\reals$, $a\in\reals$ and
$\bowtie\in\set{\geq,\leq,>,<}$, we denote $\Psi_{\bowtie a} = \set{
  x\in\Psi \st x \bowtie a }$.  For example, $\double_{>3.2} =
\set{x\in\double\st{x>3.2}}$.  The supremum of a subset of real numbers
$\Psi\subseteq\reals$ is $\sup\rb{\Psi}\in\reals\bigcup\set{\infty}$
and the infimum is $\inf\rb{\Psi}\in\reals\bigcup\set{-\infty}$.  The
absolute value of a real number $x$ is $\abs{x} = \sup\set{x,-x}$.
The real part of a complex number $z$ is $\real{z}$ and the imaginary
part is $\imag{z}$.  The absolute value of a complex number $z$ is
$\abs{z} = \rb{(\real{z})^2 + (\imag{z})^2}^{1/2}$.  For two reals $r$
and $s$, the result of multiplying $r$ and $s$ is $rs$, the result of
adding of $r$ and $s$ is $r+s$, the result of subtracting $s$ from $r$
is $r-s$ and the result of dividing $r$ by $s$ if $s\neq 0$ is
$\frac{r}{s}$.  For non-negative integer $i$ and real number $x$, the
result of raising $x$ to the power $i$ is $x^i$.  The set of $n\times
m$ matrices containing elements from a set $\Psi$ is $\Psi^{n\times
  m}$.  For simplicity, we denote $\Psi^{n\times 1}$ as $\Psi^n$.  The
set of real vectors is $\reals^n$.  The $i^{th}$ row $j^{th}$ column
element of a matrix $X$ is denoted $X_{ij}$.  Similarly, for real
vector $x$, $x_i$ denotes the $i^{th}$ component of $x$.  The
submatrix of $X$ containing rows $l_1$ to $l_2$ and columns $k_1$ to
$k_2$ is denoted $X_{l_1:l_2,k_1:k_2}$.  An $n\times m$ matrix
containing a repeated element $z$ is denoted $\sqb{z}_{n\times m}$.
The identity $n\times n$ square matrix containing all ones along its
diagonal and rest zeros is denoted $\identity{n}$.  The transpose of a
matrix $X$ is $X^T$ where $X^T_{ij} = X_{ji}$.  For
$X\in\reals^{n\times m}$ and $Y\in\reals^{m\times p}$, the matrix
product is $XY\in\reals^{n\times p}$ where $\forall
i\in\set{1,\ldots,n},\forall j\in\set{1,\ldots,p},\rb{XY}_{ij} =
\sum_{k=1}^{m}X_{ik}Y_{kj}$.  The euclidean norm of a real vector
$x\in\reals^n$ is $\norm{x} = \rb{\sum_{i=1}^n\abs{x_i}^2}^{1/2}$.
The infinity norm is $\norm{x}_\infty =
\sup\set{\abs{x_1},...,\abs{x_n}}$.  For $r\in\reals_{> 0}$ and
$x\in\reals^n$, we define the box of width $r$ around $x$ as
$\ball{x}{r} = \set{ y\in\reals^n: \norm{y - x}_\infty \leq r }$.

If $\Psi\subseteq\reals^n$, a function $f:\Psi\rightarrow\reals^m$ is
called \emph{continuous} if $\forall
x\in\Psi,\forall\epsilon\in\reals_{>0}$, there exists
$\delta\in\reals^n$ such that if $y\in\ball{x}{\delta}\bigcap\Psi$,
then $f(y)\in\ball{f(x)}{\epsilon}$.  A function
$f:\Psi\rightarrow\reals^m$ is called \emph{piecewise continuous} if
there exists a finite number of sets $\Psi_1,\ldots,\Psi_k$ such that
$\bigcup_{i=1}^k\Psi_i=\Psi$ and $f_i:\Psi_i\rightarrow
\reals^m$ where $\forall x\in\Psi_i~f_i\rb{x}=f\rb{x}$, is continuous
for all $i\in\set{1,\ldots,k}$.

A partial function $f:\reals^n\rightarrow \reals^m$ is a relation $f
\subseteq\reals^n\times\reals^m$ such that if $(x,y_1),(x,y_2)\in
f$, then $y_1=y_2 = f(x)$.  The domain of a partial function
$f:\reals^n\rightarrow\reals^m$ is $\dom{f}
= \set{x\in\reals^n\st{\exists y\in\reals^m,~(x,y)\in f}}$.  Given two
partial functions $f:\reals^n\rightarrow\reals^m$ and
$g:\reals^m\rightarrow\reals^p$, their composition is \\$f\circ g
= \set{(x,y)\in\reals^{n\times p}\st{\exists z\in\reals^m~(x,z)\in
f,~(z,y)\in g}}$.  For every $n\in\integers_{>0}$, we consider the
subset $\fe{n}$ of partial functions, defined recursively as follows.
%
\begin{enumerate}
\item $f\in\fe{n}$ if $\exists a\in\double^n,f = \set{(x,a^Tx)\st{x\in\reals^n}}$. 
\item $\forall f_1,f_2\in\fe{n}, \set{ \frac{1}{f_1}, f_1+f_2, f_1-f_2, f_1f_2}\subseteq\fe{n}$.
\item $\forall f\in\fe{n}, \set{ \sin\circ{f}, \cos\circ{f}, \exp\circ{f}, \log\circ{f} } \subseteq \fe{n}$. 
\end{enumerate}
%
We say that the \emph{symbolic derivative} of
$f\in\fe{n}$ exists and is equal to a tuple of partial functions $\nabla f = \rb{\nabla_1
f,\ldots,\nabla_n f}\in\fe{n}^n$ if all of the following are true.
%
\begin{itemize}
\item $\forall i\in\set{1,\ldots,n}$, $\dom{f}\subseteq \dom{\nabla_i f}$.
\item $\forall x\in\dom{f}$ and $\forall\epsilon\in\reals_{>0}$, there exists $r\in\reals_{>0}$ 
such that $\forall h\in\ball{0}{r}$, we have $(x+h)\in\dom{f}$ and
%
\begin{align*}
& \frac{\norm{f\rb{x+h}-\sum_{i=1}^n \nabla_if\rb{x}h_i}}{\norm{h}}
< \epsilon.~\numberthis\label{eqn:symder}
\end{align*}
%
\end{itemize}
%
\begin{example}
Let us consider a partial function $f\in\fe{2}: f(x,y) =
\frac{sin\rb{y}}{x}$.  By applying the basic rules of calculus, we get
$\frac{\partial{f}}{\partial{x}}\left|_{x,y}\right. =
\frac{-sin(y)}{x^2}$ and
$\frac{\partial{f}}{\partial{y}}\left|_{x,y}\right. =
\frac{\rb{cos(y)}}{x}$. So, the pair of partial functions $\rb{
  \rb{(x,y),\frac{-sin(y)}{x^2}}, \rb{(x,y), \frac{{cos(y)}}{x}} }$
  is a symbolic derivative of $f$.  
\end{example}
%
\emph{Remark:}  For the specific subset of partial functions $\fe{n}$
  defined above,
  if $f\in\fe{n}$, the symbolic derivative $\nabla{f}$ and the
symbolic double derivative $\nabla^2f\in\fe{n}^{n\times
n}: \rb{\nabla^2 f}_{ij} =
\nabla_j\nabla_i f$ exist.  They can be computed by using
basic rules of calculus.
%
\subsection{Use of interval arithmetic}
For $a,b\in\double$ such that $a<b$, we denote\\ $\interval{a}{b} =
\set{x\in\reals\st a\leq x\leq b}$, $[-\infty,a] = \set{x\in\reals\st
  x\leq a}$, $[a,\infty) = \set{x\in\reals\st a\leq x}$ and
  $[-\infty,\infty] = \reals$.  We denote the set of all intervals as
  $\intervals$.  For $u,v\in\intervals$, interval arithmetic
  (see~\cite{bronnimann2006design}) gives us interval bounds
  $uv,(u+v),(u-v)\in\intervals$ such that $uv\supseteq\set{xy\st x\in
    u, y\in v}$, $\rb{u+v}\supseteq\set{x+y\st x\in u, y\in v}$ and
  $\rb{u-v}\supseteq\set{x-y\st x\in u, y\in v}$.  Similarly, for a
  partial function $f\in \fe{n}$ and interval $u\in\dom{f}$, we use
  interval arithmetic to correctly bound the image of $f$ over $a$,
  denoted $\iv{f}{u} \supseteq \set{f\rb{x}\st{x\in u} }$.


The set of $n\times m$ interval matrices is $\intervals^{n\times m}$,
and $n$-dimensional interval vectors is $\intervals^{n\times 1}$,
conveniently denoted by $\intervals^{n}$.  For $A\in\reals^{n\times
m}$ and $\bar{X}\in\intervals^{n\times m}$, we say $A\in \bar{X}$ if
$\forall
\rb{i,j}\in\set{1,\ldots,n}^2, A_{ij}\subseteq \bar{X}_{ij}$.  The interval
matrix $X$ is said to be bounded if $\forall
i\in\set{1,\ldots,n}, \forall j\in\set{1,\ldots,m}, \exists
a,b\in\double: \bar{X}_{ij} = [a,b]$.  The matrix product
$\bar{X}\bar{Y}$ of two interval matrices
$\bar{X}\in\intervals^{n\times p}$ and $\bar{Y}\in\intervals^{p\times
m}$ is $\bar{X}\bar{Y} = \sum_{k=1}^p\bar{X}_{ik}\bar{Y}_{kj}$. The
sum of interval matrices $\bar{X}$ and $\bar{Y}$ is $\bar{X}
+ \bar{Y}$, where $\rb{\bar{X}+\bar{Y}}_{ij} = \bar{X}_{ij}
+ \bar{Y}_{ij}$.  It follows that if $A\in \bar{X}$ and
$B\in \bar{Y}$, then $AB\in \bar{X}\bar{Y}$ and
$A+B\in \bar{X}+\bar{Y}$.  If $X\in\double^{n\times m}$, then
$\ivmat{X}\in\intervals$ is an interval matrix such that $\forall
(i,j)\in\set{1,\ldots,n}\times\set{1,\ldots,m}, \ivmat{X}_{ij} =
[X_{ij},X_{ij}]$.  For an interval matrix $\bar{X}$, we denote $\mi{\bar{X}} =
\frac{\ivmat{\sup\rb{\bar{X}}} + \ivmat{\inf\rb{\bar{X}}}}{[2,2]}$ and $\rad{X} =
\bar{X} - \mi{\bar{X}}$.  The diagonal interval matrix containing elements of an
interval vector $u$ along its diagonal is denoted $\diag{u}$.  If
$f\in\fe{n}^{m\times p}$ is a matrix of function expressions, then
$\iv{f}{u}\in \intervals^{m\times p}$ is an interval matrix where
$\rb{\iv{f}{u}}_{ij} = \iv{f_{ij}}{u}$.  For $u,v\in\intervals$, we
denote $\meet{u}{v} =
[\sup\rb{\inf\rb{u},\inf\rb{v}},\inf\rb{\sup\rb{u},\sup\rb{v}}]$ and
$\join{u}{v} = [\inf\rb{u\bigcup v}, \sup\rb{u\bigcup v}]$.  If
$\bar{X},\bar{Y}\in\intervals^{n\times m}$, then
$\meet{\bar{X}}{\bar{Y}},\join{\bar{X}}{\bar{Y}}\in\intervals^{n\times m}$ such that
$\rb{\meet{\bar{X}}{\bar{Y}}}_{ij} = \meet{\bar{X}_{ij}}{\bar{Y}_{ij}}$ and
$\rb{\join{\bar{X}}{\bar{Y}}}_{ij} = \join{\bar{X}_{ij}}{\bar{Y}_{ij}}$.
%
In addition to the above intervals, we denote
%
$[a,b) = \set{x\in\reals\st{a\leq x<b}}$, $(a,b] = \set{x\in\reals\st{a< x\leq b}}$, $(a,b) = \set{x\in\reals\st{a< x < b}}$.


%
\section{Nonlinear system and linearization}
An $n$-dimensional nonlinear dynamical system with $m$ inputs is
specified by a tuple $\rb{f,\inpset,\Omega}$, where $f =
\rb{f_1,\ldots,f_n}\in \fe{n+m}^n$ is an
$n$-tuple from the subset of partial functions $\fe{n+m}$, an
interval vector $\inpset\in\intervals^m$ is called the \emph{input
set}, and $\Omega\in\intervals^n$ is called the \emph{state space}.
For any $T\in\reals_{>0}$, a
function \mbox{$\trfn{x}:[0,T)\rightarrow \reals^n$} is called
a \emph{state trajectory} if there exists a \emph{piecewise
continuous} function $\trfn{u}:[0,T)\rightarrow \inpset$,
called \emph{input trajectory}, such that all of the following is
true.
%
\begin{align*}
& \forall t\in[0,T), \tr{x}{t}\subseteq \Omega,\\
& \forall
t\in[0,T)\ \forall\epsilon\in\reals_{>0}, \exists\delta\in\reals_{>0}~\text{such
that}\\
& \ \ \ \  \forall h\in[-\delta,\delta]\bigcap[0,T]~\norm{ \frac{ \tr{x}{t+h}
- \tr{x}{t} }{h} - f_i\rb{ \mymatrix{ \tr{x}{t}\\ \tr{u}{t} }
}}< \epsilon~\numberthis\label{eqn:dynamics}
\end{align*}
%
In the rest of the paper, $n$ is the dimension of a nonlinear system
and $m$ is the number of inputs, unless otherwise stated.  For a
trajectory $\trfn{x}:[0,T)\rightarrow \reals^n$ and $\tau<T$, we
define $\trfn{x}|_{[0,\tau]}
= \trfn{y}:[0,T)\rightarrow \reals^n: \forall t\in[0,\tau), \tr{y}{t}
= \tr{x}{t}$.   The \emph{reachable set} of the nonlinear system $H
=\rb{f,\inpset,\Omega}$ from an initial set $\init$ at a time point
$t$, denoted $\reach{H}{t}{\init}$, is a set containing all points
$\tr{x}{t}$ where $\trfn{x}$ is a state trajectory of $H$ and
$\tr{x}{0}\in\init$.  Similarly, $\reach{H}{[t_1,t_2]}{\init}
= \bigcup_{t\in[t_1,t_2]}\reach{H}{t}{\init}$.
%
\begin{example}\label{eg:ill}
Let us denote the function expressions $\proj{1} = x$, $\proj{2} = y$,
$\proj{3} = \theta$ and $\proj{4} = u$.  A $3$-dimensional nonlinear
system with one input, state space $\reals^n$ and input set $U =
[-0.3,0.3]$ can be specified as $H = \rb{f,[-0.3,0.3],\reals^3}$ where
%
\begin{align*}
& f_1 = \rb{ u\rb{ x + \theta + 0.1y } - 1}x,
& f_2 = \rb{ u\rb{ y + \theta + 0.1x } - 1}y,\\
& f_3 = -\theta + 0.001\theta^2.
\end{align*}
%
Let us consider the zero input trajectory $\tr{u}{t} = 0~\forall
t\in[0,5)$.  In this case, two state trajectories are $\forall
t\in[0,5)$,
%
\begin{align*}
\tr{x}{t} = \mymatrix{ -0.1\exp\rb{-t}\\ 0.1\exp\rb{-t}\\ 0.1\exp\rb{-t} },~
& \tr{x}{t} = \mymatrix{ -0.2\exp\rb{-t}\\ -0.1\exp\rb{-t}\\ -0.1\exp\rb{-t} }
\end{align*}
%
which begin at $\tr{x}{0} = \mymatrix{-0.1,0.1,0.1}^T$ and $\tr{x}{0}
= \mymatrix{-0.2,-0.3,-0.1}^T$, respectively.
%
But there are also state trajectories without an analytic form.  The
objective of flowpipe computation is to compute bounds along different
directions on the reachable set $\reach{H}{t}{\init}$ at each time
point in $[0,T)$, which originate $\init$ (eg. see Fig~\ref{fig:ill}).
\end{example}
%
%%  Computing the exact
%% reachable set of nonlinear systems is computationally infeasible in
%% general because nonlinear ODEs may not have analytically closed form
%% solutions.  Instead, an overapproximation of the reachable set at each
%% time point, called \emph{flowpipe}, can be computed to verify
%% essential reachability properties like safety and stability.  The
%% flowpipe approximation of reachable set is
%% represented in a computer by a data structure which can be manipulated
%% efficiently to check reachability properties.  Such data structures
%% are called set representations, whose examples
%% are \emph{boxes}, \emph{polytopes}, \emph{zonotopes}, \emph{Taylor
%% models}, \emph{barrier certificates} and \emph{polynomial
%% zonotopes}~(\todo{cite all}).
%
\paragraph{Linearization:}  We say that a dynamical system
$L = \rb{g,U\times V,\Omega}$, where $U\in\intervals^m$ and
$V\in\intervals^n$ are both input sets and $g\in\fe{2n+m}^n$, is
linear if we have matrices $A\in\reals^{n\times n}$ and
$B\in\reals^{m\times n}$ such that for all $i\in\set{1,\ldots,n}$,
$x\in\reals^n$, $u\in U$ and $v\in V$, we have $ g_i\rb{\mymatrix{x^T
& u^T & v^T}^T} = \rb{Ax + Bu + v}_i$.  For convenience, we also
denote the above linear system as $L = \rb{A,B,U,V,\Omega}$.  This
linear system is said to overapproximate a nonlinear system $H
= \rb{f,\inpset,\Omega}$, denoted $L\succeq H$, if for all
$x\in\Omega$ and $u\in\inpset$, we have $f\rb{\mymatrix{x^T & u^T}}^T
- Ax - Bu \in V$.  In this case, the reachable set of $H$ is a subset
of the reachable set of $L$ at every time point as explained in the
following lemma.
%
\begin{lemma}\label{lem:inclin}
Let us consider that a linear system $L = \rb{A,B,U,V,\Omega}$ is an
overapproximation of a nonlinear system $H = \rb{f,U,\Omega}$.  Then
for all $t\in[0,\infty]$ and $\init\subseteq\Omega$, we have
$\reach{H}{t}{\init}\subseteq \reach{L}{t}{\init}$.
\end{lemma}
%
\begin{proof}
See appendix.
\end{proof}
%
There are efficient algorithms to compute flowpipes of linear systems
with good accuracy~\cite{girard2005reachability,girard2008efficient}.
So, it would be useful to overapproximate a nonlinear system by a
linear system while computing flowpipe, as follows.
%
\begin{lemma}[Linear overapproximation]\label{lem:linearization}
Let us consider a bounded interval vector $\Omega\in\intervals^n$ and
a nonlinear system $H = \rb{f,U,\Omega}$ where $f\in\fe{n+m}^n$ and
$U\in\intervals^m$ is a bounded interval vector.  Let us define $W
= \mymatrix{\Omega^T & U^T}^T$,
\begin{align*}
\begin{split}
\stmat{f}{\Omega}{U} =  \iv{{\nabla{f}_{1:n,1:n}}}{\mi{W} },
%
~\inpmat{f}{\Omega}{U}
= \iv{{\nabla{f}_{1:n,n+1:n+m}}}{ \mi{W} },
\end{split}\\
%
\begin{split}
& \rb{\taylor{f}{\Omega}{U}}_i
= \frac{1}{2}\rad{W}\rb{{\nabla^2f_i}\rb{W}}\rad{W}~\%
% = \frac{1}{2}\sum_{k=1}^{n+m}\sum_{j=1}^{n+m}\iv{\nabla_k\nabla_j{f_i}}{{W}}\rad{W_k}\rad{W_j},~\%
\text{Taylor err.}\\
%
& \gamma\rb{f,\Omega,U} = 2\rad{ \stmat{f}{\Omega}{U}\Omega +
\inpmat{f}{\Omega}{U}U
},~~~\%\text{Floating point error bounds}
\end{split}\\
%
\begin{split}
\err{f}{\Omega}{U} = \iv{f}{ \mi{W} }
+ \taylor{f}{\Omega}{U} + \gamma\rb{f,\Omega,U}
- \stmat{f}{\Omega}{U}\mi{\Omega} - \inpmat{f}{\Omega}{U}\mi{U}.
\end{split}~\numberthis\label{eqn:linearize}
\end{align*}
Then $
H \preceq
\rb{\inf\rb{\stmat{f}{\Omega}{U}},
\inf\rb{\inpmat{f}{\Omega}{U}},
U,
\err{f}{\Omega}{U}
} $
%
\end{lemma}
%
\begin{proof}
Based on Taylor remainder theorem for expansion around the center
$\mi{W}$, for all $i\in\set{1,\ldots,n}$ and $z\in W$, there exists
$r\in W$ such that $f_i(z) = $
%
\begin{align*}
& f_i\rb{\mi{W}} + \nabla f_i\rb{\mi{W}}\rb{z - \mi{W}}
+ \frac{1}{2}\rad{W}\rb{{\nabla^2f_i}\rb{W}}\rad{W}~\numberthis\label{eqn:taylor}
\end{align*}
%
The lemma follows by substituting both $r$ and $z$ by
the interval vector $W$ in the R.H.S of Equation~(\ref{eqn:taylor}) and applying
interval arithmetic.
\end{proof}
%
\begin{example}
Let us consider the nonlinear system $H = \rb{f,[-0.2,0.2],[-1,1]^3}$
where $f$ is the tuple specified in Example~\ref{eg:ill}.  By applying
Lemma~\ref{lem:linearization}, we get a linear overapproximation $L
= \rb{A,B,[-0.2,0.2],V,[-1,1]^3}\succeq H$ where we illustrate below
how $A_{11}$, $B_{11}$, and $V_{1}$ are computed.
%
\begin{align*}
& A_{11} = \inf\rb{\rb{\frac{df_1}{dx}}\mid_{\rb{x,y,z,u} =
\ivmat{(0,0,0,0)}}} \\ & = \inf\rb{\rb{u\rb{ x + \theta + 0.1y } - 1 +
xu}\mid_{\rb{x,y,z,u} = \ivmat{(0,0,0,0)}}} = -1 \\
%
& B_{11} = \inf\rb{\rb{\frac{df_1}{du}}\mid_{\rb{x,y,z,u} =
\ivmat{(0,0,0,0)}}} \\ &= \inf\rb{\rb{x\rb{ x + \theta + 0.1y
}}\mid_{\rb{x,y,z,u} = \ivmat{(0,0,0,0)}}} = 0 \\
%
& \nabla_1\nabla_2f_1\rb{W_1-0}\rb{W_2-0} =
0.1*[-0.2,0.2]*[-1,1]*[-1,1],~\text{likewise},\\
& \taylor{f}{[-1,1]^3}{[-0.2,0.2]}
= \sum_{k=1}^4\sum_{j=1}^4\nabla_k\nabla_jf_1\rb{W_j-0}\rb{W_k-0}
%% & = 1/2*(2*[-0.2,0.2]*[-1,1]*[-1,1] + 0.1*[-0.2,0.2]*[-1,1]*[-1,1] +\\
%% & [-0.2,0.2]*[-1,1]*[-1,1] + \\
%% &  \rb{[-1,1]+[-1,1]+[0.1,0.1]*[-1,1]}*[-0.2,0.2]*[-1,1] + \\ 
%% & 0.1*[-0.2,0.2]*[-1,1]*[-1,1] + 0 + 0 + 0 + \\ 
%% & [-0.2,0.2]*[-1,1]*[-1,1] + 0 + 0 + 0 + \\
%% & \rb{[-1,1]+[-1,1]+[0.1,0.1]*[-1,1]}*[-0.2,0.2]*[-1,1] + 0 + 0 + 0) =\\
= [-1.26, 1.26]\\
%
& V_1 = \rb{\iv{f}{ 0 }
+ \taylor{f}{[-1,1]^3}{[-0.2,0.2]}
- A*0 - B*0}_1 \\
& = 0 + [-1.26,1.26] + 0 + 0 = [-1.26, 1.26]
\end{align*}
%
\end{example}
%
\paragraph{Reach set approximation of linear system:}  The reachable
set of a linear system $L = \rb{A,B,U,V,\Omega}$ in a time interval
$[t_1,t_2]$ has the following overapproximation.
%
\begin{lemma}\label{lem:linreach}
Let us consider
\begin{align*}
& \dismat{A}{L}{[t_1,t_2]} = \ivmat{\identity{n}}
+ \ivmat{A}[t_1,t_2]
+ \ivmat{A}{[t_1,t_2]}{[t_1,t_2]}
+ \ivmat{A}[0,t_2][0,t_2][0,t_2]\\
& \dismat{B}{L}{[t_1,t_2]}
= \ivmat{B}{[t_1,t_2]}
+ \ivmat{B}\ivmat{A}[0,t_2][0,t_2]\\
& \dismat{C}{L}{[t_1,t_2]}
= \ivmat{\identity{n}}{[t_1,t_2]}
+ \ivmat{\identity{n}}\ivmat{A}{[0,t_2]}[0,t_2].\\
%
& \text{Then},~\reach{L}{[t_1,t_2]}{\init} \subseteq \set{\dismat{A}{L}{[t_1,t_2]}x
+ \dismat{B}{L}{[t_1,t_2]}u + \dismat{C}{L}{[t_1,t_2]}v\st x\in \init, u\in U, v\in
V}.~\numberthis\label{eqn:linreach}
\end{align*}
\end{lemma}
%
\begin{proof}
See appendix.
\end{proof}
%
We can compute a crude interval overapproximation of the reachable set
$\reach{H}{[0,t]}{\init}$ in the interval $[0,t]$, as follows.  But we
can compute more accurate bounds by using advanced procedures which make
use of the below result as a subprocedure.
%
\begin{lemma}~\label{lem:bloat}
Let us consider a nonlinear system $H = \rb{f,U,\reals^n}$,
$\epsilon\in[0,\infty)$, a bounded interval vector
$\bounds^{0}\in\intervals^n$, and a sequence of interval vectors
$\rb{\bounds^i}_{i=0}^\infty$ where
%
\begin{align*}
& L_i = \rb{ \inf\rb{\stmat{f}{d^i}{U}},
\inf\rb{\inpmat{f}{d^i}{U}},U,\err{f}{\rb{\bounds^i+[-\epsilon,\epsilon]^n}}{U} }\\
& \bounds^{i+1}
= \join{\rb{\bounds^i+[-\epsilon,\epsilon]^n}}{\rb{\dismat{A}{L_i}{[0,t]}\bounds^0
+ \dismat{B}{L_i}{[0,t]}U + \dismat{C}{L_i}{[0,t]}V}}.~\numberthis\label{eqn:iter}
\end{align*}
%
For $K\in\integers_{>0}$, let $\ivflow{H}{t}{K}{\epsilon}{\bounds^0}
= \begin{cases}\bounds^{K}
& \text{if}~\rb{\dismat{A}{L_i}{[0,t]}\bounds^0
+ \dismat{B}{L}{[0,t]}U + \dismat{C}{L}{[0,t]}V} \subseteq
\bounds^K\\ \reals^n & otherwise \end{cases}$.  Then,
$\ivflow{H}{t}{K}{\epsilon}{\bounds^0}\supseteq\reach{H}{[0,t]}{\bounds^0}$.
\end{lemma}
%
%Proof of above lemma is in the appendix.
\begin{proof}
See appendix.
\end{proof}

%
\section{Intersection of unions flowpipe computation}
\is{Do we need a couple of introductory sentences to justify why we are concentrating on the technique proposed in [2]?}

\emph{Review:}  In the
linearization based flowpipe computation proposed by Althoff
et. al.~\cite{althoff2008reachability}, at any time $t$ where
$\tr{X}{t} = \init$, first a linear system overapproximation
$\rb{\inf\rb{\stmat{f}{\Omega}{U}},
\inf\rb{\inpmat{f}{\Omega}{U}},U,\err{f}{\Omega}{U}}$  of the
nonlinear system $H$ is computed in a neighborhood $\Omega$ of
$\init$.  The set $\Omega$ is computed as a crude over-approximation
of the reachable set between $[0,t+\delta]$ for a small $\delta$, like
the interval over-approximation in Lemma~\ref{lem:bloat}.  Then, a
more accurate flowpipe is computed based
on the linearized dynamics.  Therefore, the accuracy of flowpipe
$\tr{X}{[t,t+\delta]}$ is positively correlated with the Taylor error
in linearization $\taylor{f}{\Omega}{U}$ in the region $\Omega$.
Since $\Omega$ is a small neighborhood of $\init$, this Taylor error
is close to \is{closely?} correlated with the Taylor error
$\taylor{f}{\overline{\init}}{U}$ inside an interval
overapproximation $\overline{\init}$ of $\init$.  The Taylor error
inside $\overline{\init}$ is positively correlated with the bounds of
$\overline{\init}$ based an Equation in~\ref{eqn:linearize}.
Therefore,~\cite{althoff2008reachability} proposed to reduce the
linearization error by dividing the reachable set $\init$ into smaller
sets based on a performance index and computing the next flowpipe as
the union of flowpipes.  However, the union method
in~\cite{althoff2008reachability} has an important drawback, explained
below.  We address this drawback by using intersection of unions
instead of just the union.
%
\subsection{Motivation for intersection of unions}
\is{I think this subsection should start from the beginning of this section.}
In Althoff. et. al~\cite{althoff2008reachability}, a reachable set
$\init$ is divided into two sets $\init_1$ and $\init_2$ which
minimizes the following quantity for a user chosen normalization error
$\rho\in\reals^n_{>0}$.
%
\begin{align*}
\sup_{i=1}^n\frac{{\sup{\taylor{f}{\overline{\init_1}}{U}}}_i}{\rho_i}\sup_{i=1}^n\frac{\sup{\taylor{f}{\overline{\Omega_2}}{U}}_i}{\rho_i}.~~\text{~\cite{althoff2008reachability}}~\numberthis\label{eqn:pi}
\end{align*}
%
For obtaining smaller divisions, the above pairwise division is used
recursively.

\emph{Drawback:}  Since the Taylor error
$\taylor{f}{\overline{\init}}{U}$ is a multi-dimensional vector, minimizing its
bounds $\sup{\taylor{f}{\overline{\init}}{U}}$ is a multi-objective optimization
problem which does not have a single best solution.  But the above
performance index is a one dimensional measure of the overall
linearization error.  Minimizing this one dimensional performance
index does not necessarily reduce signifcantly the linearization error
along all coordinates.  For illustration, let us consider the
nonlinear system in Example~\ref{eg:ill}.  Let $\init =
[-1,1]\times[-0.5,0.5]\times[-0.1,0.1]$ be the reachable set at a time
point and we take the linearization error before splitting $\rho
= \taylor{f}{\init}{U}$ as the normalization error for splitting.
Using the above performance index, we get that $\init$ should be split
along $x$-axis into $\init_1 = [-1,0]\times[-0.5,0.5]\times[-0.1,0.1]$
and $\init_2 = [0,1]\times[-0.5,0.5]\times[-0.1,0.1]$.  However, there
will be very little reduction in the linearization error along
$y$-axis, which least depends on the width along $x$-axis due to the
multiplication coefficient $0.1$ of $xy$ in $f_3$.  Instead, the best
division to minimize the $y$-directional linearization error is to
divide along $y$-coordinate as $\init_1 =
[-1,1]\times[-0.5,0]\times[-0.1,0.1]$ and $\init_2 =
[-1,1]\times[0,0.5]\times[-0.1,0.1]$.  This means reducing the
linearization error using the above performance index need not
significantly reduce the linearization error along all coordinates.
Besides, it is not known how to choose a normalization error $\rho$
for good flowpipe accuracy.

\paragraph{New method:}  In this paper, we propose an better approach
that ensures reduction in the linearization error along each
individual direction in a finite set of directions.  To do so, we
use \emph{intersection of unions} (IoU).  For each direction, we find
an optimized division of the reach set for reducing the linearization
error.  Then we intersect the union sets corresponding to the
optimized division for linearization error along each direction.  This
way, we ensure that the linearization error is significantly reduced
along each direction in the given set of directions.

As a heuristic, we choose the directions as the eigenvectors of the
nonlinear system at the center of the initial set.  Furthermore, we
only divide the initial set because of two reasons.
%
\begin{enumerate}
\item Dividing a reach set increases the representation complexity due to
increase in the number of elements in the union.  So, iterative
division of reach set at various time stamps will blow up the
representation complexity.
\item  A non-interval flowpipe at some time stamp cannot
be divided accurately and reduces the accuracy of future flowpipe.
Flowpipes at future time points are better approximated as
non-intervals.  However, since we consider the initial set as an interval, the initial set can be divided accurately.
\end{enumerate}
%
%% Henceforth, we only divide the interval initial set to
%% compute an intersection of unions representation.
%
\subsection{Computing set of optimized division vectors}
Recall that our initial set is the interval denoted $X_0$.  Let us
denote the matrix of eigenvectors of $\inf\rb{\stmat{f}{X_0}{U}}$ as
$\eig$.  Our objective is to minimize the projection of the Taylor
error along each column vector of $\eig$.  For each direction, we have
a optimized way of dividing the set $X_0$ denoted by an integer vector
$q$ that represents the number of divisions along each dimension.  So,
for the set of eigenvector, we get a set of optimized division
vectors.  Subsequently, we cast the initial set as intersection of
union of the divisions corresponding to different optimized division
vectors.  For convenience, let us denote $\bounds=X_0$.  Given a
vector of positive integer values $q\in\integers_{>0}^n$,
called \emph{division vector}, we define a set of interval vectors
whose union is a close overapproximation of $\bounds$ with only
floating point approximation error, as follows.
%
\begin{align*}
D_q
= & \left\{r_1\times...\times r_n\right.\\
& \left.\st{\forall i\in\set{1,\ldots,n}~\exists k_i\in\set{1,\ldots,q_i}}:~r_i
= \ivmat{\inf\rb{\bounds_i}}+k_i\frac{\rad{\bounds}_i}{q_i}\right\}.
\end{align*}
%
For all intervals in $D_q$, an vector upper bound
$\utaylor{f}{U}{\bounds}{q}$ on the magnitude of Taylor error is given below.
%
\begin{align*}
\rb{\utaylor{f}{U}{\bounds}{q}}_i
= \sup\rb{\mymatrix{\rad{d}\rb{\diag{q}}^{-1}\\ \rad{U}}^T{\nabla^2
f_i}\rb{\mymatrix{d\\ U}}\mymatrix{\rad{d}\rb{\diag{q}}^{-1}\\ \rad{U}}}
\end{align*}
%
For any $k\in\set{1,\ldots,n}$, the projection of the  bound
along the $k^{th}$ eigenvector is given by
$\utaylor{f}{U}{\bounds}{q}\eig_{:,k}
= \utaylor{f}{U}{\bounds}{q}\real{\eig_{:,k}}
+ \iota\utaylor{f}{U}{\bounds}{q}\imag{\eig_{:,i}}$.  We select $q$ to
minimize the maximum absolute value of this bound, by using the
following greedy optimization.  Let us consider for any
$i\in\set{1,\ldots,n}$, \mbox{$\omega\rb{q,i}\in\integers^n$}, where $\omega\rb{q,i}_j
= \begin{cases}2q_j & j=i\\ q_j & i\neq j \end{cases}$.

\begin{algorithm}
\caption{Optimizing division vector for $k^{th}$
eigenvector} $q\gets \mymatrix{1,\ldots,1}^T\in\integers^n$\;
\While{$\prod_{j=1}^nq_j<2^\eta$}{
$ind \gets \argmin_{i=1}^n\begin{cases}\rb{\sup\rb{\real{\utaylor{f}{U}{\bounds}{\omega\rb{q,i}}\eig_{:,k}}}}^2\\
 + \rb{\sup\rb{\imag{\utaylor{f}{U}{\bounds}{\omega\rb{q,i}}\eig_{:,k}}}}^2\end{cases}$\;
$q\gets \omega\rb{q,ind}$
}
\end{algorithm}
%
Subsequently, we get a set of optimized division vectors corresponding
to different eigenvectors.  We denote this set as as $\dopt{\eta}$ \is{could we use a different symbol? $\dopt{}$ looks like asymptotic complexity.},
where $2^\eta$ is the maximum number of divisions.
%
\begin{example}
Let us consider the nonlinear system $H$ in Example~\ref{eg:ill}, an
interval vector $h = [0,1]^3$ and maximum number of divisions $\eta =
4$.  The eigenvectors at the origin are the coordinate vectors
$\mymatrix{1 & 0 & 0}^T$, $\mymatrix{0 & 1 & 0}^T$ and $\mymatrix{0 &
0 & 1}^T$.  To reduce the linearization error along $\mymatrix{1 & 0 &
0}^T$, the best possible division vector is $\mymatrix{2 & 1 & 2}^T$,
i.e., divide $x$ and $\theta$ coordinates.  For the direction
$\mymatrix{0 & 1 &0}^T$, the optimum division vector is $\mymatrix{1 &
2 & 2}^T$, and for the direction $\mymatrix{0 & 0 & 1}^T$, the optimum
division vector is $\mymatrix{1 & 1 & 4}^T$.  So, $\dopt{4}
= \set{\mymatrix{2 & 1 & 2}^T, \mymatrix{1 & 2 & 2}^T, \mymatrix{1 & 1
& 4}^T}$.  The three different division vectors results three
different kinds of divisions of $[0,1]^3$ whose intersection is
represented as an IoU of interval zonotopes.
\end{example}
%
\subsection{Casting initial set as intersection of unions of interval zonotopes}
For different division vectors, we get different types of unions which
overapproximate the initial set.  So, we can represent the initial set
as intersection of unions (IoU) of intervals.  However, intervals do
not provide good approximation of reachable sets at future time
stamps.  Alternatively, \emph{zonotopes} are known to provide good
approximation of reachable sets when linearization is
used~\cite{todo}.  Then, since our initial set is an intersection of
unions, for using zonotopes we need to recast it as intersection of
union of zonotopes. However, it is known that zonotopes do not provide
good approximation of intersection between them~\cite{todo}, which is
required in our IoU representation.  Therefore, we generalize
zonotopes to interval zonotopes to compute better approximation of
intersection between zonotopes and also soundly approximate floating
point computations.  It is defined as follows.
%
\begin{definition}[Interval Zonotope]
Let us consider $l\in\integers_{>0}$, $\gen\in\intervals^{n\times
ln},\cen\in\intervals^{n}$ and $\bounds\in\intervals^n$.  An interval
zonotope of order $l>1$ is the tuple $\rb{\gen,\cen,\bounds}$ which
represents the following set.
%
\begin{align*}
\iz{\gen}{\cen}{\bounds}
= \set{x\in\reals^n \st{\exists \zeta\in[-1,1]^{nl}:~x\in\bounds,~x \in \gen\ivmat{\zeta}+\cen} }
\end{align*}
%
If $\zt = \iz{\gen}{\cen}{\bounds}$, then we represent $\overline{\zt}
= \bounds$.
\end{definition}
%
Before computing the IoU interval zonotope representation of the
initial set, we will discuss a transformation operation on the interval zonotope
that will be used in the IoU recasting and flowpipe computation.
%
\begin{lemma}[Linear transformation and sum]\label{lem:lintrans}
Let us consider $A\in\intervals^{n\times n}$, $w\in\intervals^n$ and an
interval zonotope $\rb{{\gen},{\cen},{\bounds}}$ of order $l>1$.  Let us
consider $\gen^\prime\in\reals^{n\times nl}$ where 
%
\begin{align*}
& \gen^\prime_{{:,n+1:ln}} = A\gen_{\rb{:,1:(l-1)n}}\\
& \gen^\prime_{{:,1:n}}
= \diag{\sup\rb{\rb{A\gen_{\rb{:,(l-1)n+1:ln}} }[-1,1]^n
+ \rad{w} }}\\
& z = \rb{A\gen}[-1,1]^n + A\cen + w.\\
& \text{Then},~\set{A^\prime x + y\st{x\in\iz{\gen}{\cen}{\bounds}, y\in\bounds,
A^\prime\in A}} \\
& \subseteq \iz{\gen^\prime}{A\cen+\mi{w}}{\meet{\rb{A\bounds + A\cen 
+ w}}{z}}~\numberthis\label{eqn:linMin}
\end{align*}
%
\end{lemma}
%
\begin{proof}
Let us consider $x\in \iz{\gen}{\cen}{\bounds}$ and $y\in w$.  So,
there exists $\zeta\in [-1,1]^n$ such that $x \in \gen\zeta + \cen$ and
also $x\in\bounds$.  Then,
%
\begin{align*}
& Ax + y \in A\gen\zeta + A\cen + y\\
& = A\gen_{{:,1:(l-1)n}}\zeta_{{:,1:(l-1)n}} +
A\gen_{:,{(l-1)n+1:ln}}\zeta_{{(l-1)n+1:ln}} + A\cen + y\\
& \subseteq A\gen_{{:,1:(l-1)n}}\zeta_{{:,1:(l-1)n}} 
 + A\gen_{:,{(l-1)n+1:ln}}[-1,1]^n + A\cen + w\\
& = \gen^\prime_{{:,n+1:ln}}\zeta_{{:,1:(l-1)n}} +
\rb{A\gen_{:,{(l-1)n+1:ln}}[-1,1]^n + \rad{w}} + A\cen +  \mi{w} 
\end{align*}
%
We have $A\gen_{:,{(l-1)n+1:ln}}[-1,1]^n + \rad{w} = \set{\gen^\prime_{:,1:n}\zeta^\prime\st{\zeta^\prime\in[-1,1]^n}}$
So, there exists $\zeta^\prime \in[-1,1]^n$ such that
%
\begin{align*}
& Ax + y =  \gen^\prime_{{:,n+1:ln}}\zeta_{{:,1:(l-1)n}}
+ \gen^\prime_{:,1:n}\zeta^\prime + A\cen + \mi{w}\\
& = \gen\mymatrix{\zeta_{{:,1:(l-1)n}}\\\zeta^\prime} + A\cen + \mi{w}~\numberthis~\label{eqn:pr1}
\end{align*}
%
Also, we get the following two bounds.
%
\begin{align*}
 & Ax + y \in A\gen[-1,1]^n + A\cen + w = z~\numberthis\label{eqn:pr2}\\
 & Ax + y \in A\bounds + A\cen + w~\numberthis\label{eqn:pr3}
\end{align*}
%
By~(\ref{eqn:pr1}),(\ref{eqn:pr2}) and (\ref{eqn:pr3}), we get that
$Ax+y\in\iz{\gen^\prime}{A\cen+\mi{w}}{\meet{\rb{A\bounds + A\cen + w}}{z}}$
\end{proof}
%
For convenience, if $\zt = \iz{\gen}{\cen}{\bounds}$, we denote the
R.H.S of~(\ref{eqn:linMin}) as $\lin{\zt}{A}{w}$.  The $n$-dimensional
interval zonotope of order $l$ which is equivalent to zero is denoted
$\zerozon{n}{l}$.
%
Based on the above lemma, for a division vector $q\in\dopt{\eta}$, we
can closely overapproximate $X_0$ as a union of interval zonotopes of
order $l>0$ as $\bigcup_{b\in
D_q}\lin{\zerozon{n}{l}}{\sqb{0}_{n\times n}}{b}$ with only floating
point overapproximation error.  To reduce the Taylor error along each
eigenvector, we store the union corresponding to each optimized
division vector in $\dopt{\eta}$ and take their intersection.  The
resulting intersection of unions is
%
\begin{align*}
\bigcap_{q\in\dopt{\eta}}\bigcup_{b\in
D_q}\lin{\zerozon{n}{l}}{\sqb{0}_{n\times n}}{b}~\numberthis\label{eqn:iouopt}
\end{align*}
%
In a computer, we represent an intersection of unions of interval zonotope sets as above with a matrix of interval zonotope tuples, called \emph{IoU interval zonotope}.
%
\begin{definition}[IoU interval zonotope]
An IoU zonotope $\zt$ is a matrix of interval zonotopes which
represents the set
%
\[
\ztset{\zt} = \bigcap_{i=1}^{\rows{\zt}}\bigcup_{i=1}^{\cols{\zt}}\ztset{\zt_{ij}}.
\]
%
The overall interval bounds on the IoU interval zonotope is
$\overline{\zt}
= \bigvee_{i=1}^{\rows{\zt}}\bigwedge_{i=1}^{\cols{\zt}}\overline{\zt_{ij}}$.
\end{definition}
%
So, we represent the optimized intersection of unions in
Equation~\ref{eqn:iouopt} by an IoU interval zonotope
$\iouopt{\eta}{X_0}$ that is assigned by the
steps~\ref{algstep:assign1}-\ref{algstep:assign2} of Algorithm~\ref{alg:main}.
%% %
%% \begin{algorithm}\caption{Initial IoU zonotope assignment}\label{alg:assign}
%% $i\gets 1$,~$j\gets 1$\;
%% \For{$q\in\dopt{q}$}{
%% \For{$b\in D_q$}{
%% $\rb{\iouopt{\eta}{X_0}}_{ij} \gets \lin{\zerozon{n}{l}}{\sqb{0}_{n\times n}}{b}$\;
%% $j\gets j+1$\; } $i\gets i+1$\; }
%% \end{algorithm}
%% %
\subsection{Computing IoU flowpipe}
To compute the reachable set at time $t$ from an interval zonotope
$\zt$, we do the following.  Let $\Omega
= \ivflow{H}{t+\epsilon}{K}{\epsilon}{\zt}$.  From above lemma, we
have $\Omega\supseteq\reach{H}{[0,t+\epsilon]}{\ztset{\zt}}$.  Then we
compute a linear overapproximation of $\rb{f,U,\Omega}$ as\\ $L
= \rb{\inf\rb{\stmat{f}{\Omega}{U}},
\inf\rb{\inpmat{f}{\Omega}{U}},
U,
\err{f}{\Omega}{U}
}$ based on Lemma~\ref{lem:linearization}.  Next we compute the linear transformation
%
\[
\flow{H}{t}{K}{\epsilon}{\zt}
= \lin{\zt}{ \dismat{A}{L}{[t,t]}}{\dismat{B}{L}{[t,t]}U
+ \dismat{C}{L}{[t,t]}V }
\]
%
The following lemma states that $\ztset{\flow{H}{t}{K}{\epsilon}{\zt}
}$ is an overapproximation of the reachable set
$\reach{H}{t}{\ztset{\zt}}$.
%
\begin{lemma}~\label{lem:reachnonlin}
Let $H = \rb{f,U,\reals^n}$ be a nonlinear system and $\zt$ an
interval zonotope.  Then
$\reach{H}{t}{\ztset{\zt}} \subseteq \ztset{\flow{H}{t}{K}{\epsilon}{\zt}}$.
\end{lemma}
%
\begin{proof}
Let $\trfn{x}:[0,T)\rightarrow \reals^n$ be a trajectory of $H$ such
that $\tr{x}{0}\in\ztset{\zt}$.  Let $\Omega
= \ztset{\ivflow{H}{t+\epsilon}{K}{\epsilon}{\zt}}$.  By
Lemma~\ref{lem:bloat}, we get $\Omega\supseteq\reach{H}{[0,t+\epsilon]}{\zt}$.  So,
$\trfn{y} = \trfn{x}|_{[0,(t+\epsilon)]}$ is a trajectory of $H^\prime
= \rb{f,U,\Omega}$.  By Lemma~\ref{lem:linearization}, we get that\\ $L
= \rb{\inf\rb{\stmat{f}{\Omega}{U}},
\inf\rb{\inpmat{f}{\Omega}{U}},
U,
\err{f}{\Omega}{U}
}$ is an overapproximation of $H^\prime$ .  By Lemma~\ref{lem:inclin},
we get
$\reach{L}{t}{\ztset{\zt}}\supseteq\reach{H^\prime}{t}{\Omega}$.  By
Lemma~\ref{lem:lintrans} and~\ref{lem:linreach}, we have
$\ztset{\flow{H}{t}{K}{\epsilon}{\zt}} \supseteq \reach{L}{t}{\ztset{\zt}}$.
So,
$\flow{H}{t}{K}{\epsilon}{\zt}\supseteq\reach{H^\prime}{t}{\Omega}$.
We showed that $\trfn{y}$ is a trajectory of $H^\prime$. Also,
$\tr{y}{t} =\tr{x}{t}$ as $t\in[0,t+\epsilon)$.  So, by the previous
inequality we get $\tr{y}{t}
= \tr{x}{t}\in\ztset{\flow{H}{t}{K}{\epsilon}{\zt}}$.  As this is true
for all trajectories $\trfn{x}$ of $H$, we have
$\reach{H}{t}{\ztset{\zt}}\subseteq\flow{H}{t}{K}{\epsilon}{\zt}$.
\end{proof}
%
Then for an IoU interval zonotope, we can compute the reachable
set at time $t$ according to the following lemma.
%
\begin{lemma}~\label{lem:ioureach}
Let us consider an IoU of interval zonotopes $\zt$,
$K\in\integers_{\geq 0}$, $\epsilon\in\reals_{>0}$.  Let us consider
an IoU interval zonotope $\flow{H}{t}{K}{\epsilon}{\zt}$ and an
interval $\ivflow{H}{t}{K}{\epsilon}{\zt}$ defined as
%
\begin{align*}
\rb{\flow{H}{t}{K}{\epsilon}{\zt}}_{ij}
= \flow{H}{t}{K}{\epsilon}{\zt_{ij}},~~~
\rb{\ivflow{H}{t}{K}{\epsilon}{\zt}}
= \bigvee_{i=1}^{\rows{\iouopt{\eta}{X_0}}}\bigwedge_{i=1}^{\cols{\iouopt{\eta}{X_0}}}\ivflow{H}{t}{K}{\epsilon}{\overline{\zt_{ij}}}.
\end{align*}
%
Then
$\reach{H}{t}{\zt}\subseteq \ztset{\flow{H}{t}{K}{\epsilon}{\zt}}$ and
$\reach{H}{[0,t]}{\zt}\subseteq \ivflow{H}{t}{K}{\epsilon}{\zt}$.
\end{lemma}
%
\begin{proof}
Let us consider a trajectory $\trfn{x}$ such that $\tr{x}{0}\in\ztset{\zt}$.
This means $\forall i\in\set{1,\ldots,N} \exists
j_i:\tr{x}{0}\in\ztset{\zt_{ij_i}}$.  By Lemma~\ref{lem:reachnonlin}, we get
that $\forall
i\in\set{1,\ldots,N}~\tr{x}{t}\in\ztset{\flow{H}{t}{K}{\epsilon}{\ztset{\zt_{ij_i}}}}$
and by Lemma~\ref{lem:bloat} we get that $\forall t\in[0,t]\forall
i\in\set{1,\ldots,N}~\tr{x}{t}\in\ivflow{H}{t}{K}{\epsilon}{\ztset{\zt_{ij_i}}}$.
This proves the lemma.
\end{proof}
%
For an IoU interval zonotope, we can refine the interval bounds in
each of its constituting elements as follows. 
%
\begin{lemma}~\label{lem:refine}
Let $\zt$ be an IoU interval zonotope where $\zt_{ij}
= \rb{\gen^{ij},\cen^{ij},\bounds^{ij}}$.  Let us consider $\widehat{\bounds}
= \bigwedge_{i=1}^{\rows{\zt}}\bigvee_{j=1}^{\cols{\zt}}\bounds^{ij}$.  We
define a refined IoU interval zonotope $\refine\rb{\zt}$ as
%
\[
\rb{\refine\rb{\zt}}_{ij}
= \rb{\gen^{ij},{\cen^{ij}},{\meet{\bounds^{ij}}
{\widehat{\bounds}}}}.
\]
%
Then $\ztset{\zt} = \ztset{\refine\rb{\zt}}$.
\end{lemma}
%
The above lemma is straighforward to derive.  In this refined
representation, we reduce the interval bounds in each constitution
element of the IoU based on the overall interval bounds $\widehat{\bounds}$
given above.

Now we have the results to compute an optimized IoU zonotope flowpipe.
The flowpipe computation is described in Algorithm~\ref{alg:main}.
Each iteration in step~\ref{step:1} of the for loops can be run independently
in parallel.  The correctness of the algorithm is proved in
Theorem~\ref{thm:main}.
%
\begin{theorem}~\label{thm:main}
Let us consider the candidate flowpipe $\trfn{X}$ computed in
Algorithm~\ref{alg:main}.  For all $t\in[0,T]$,
$\reach{H}{t}{X_0}\subseteq \ztset{\tr{X}{t}}$.
\end{theorem}
%
\begin{proof}
We prove the theorem inductively.

\paragraph{Claim:}  Let us consider that for some
$i\in\integers_{\geq 0}:i\delta<T$, we have
$\reach{H}{i\delta}{X_0}\subseteq\tr{X}{i\delta}$.  Then $\forall
t\in[i\delta,(i+1)\delta]~\reach{H}{t}{X_0}\subseteq \ztset{\tr{X}{t}}$.

\paragraph{Proof of claim:}
By Lemmas~\ref{lem:ioureach} and~\ref{lem:refine}, we get
$\reach{H}{\delta}{\tr{X}{i\delta}}\subseteq \ztset{\flow{H}{\delta}{K}{\epsilon}{\tr{X}{i\delta}}}$
and\\
$\reach{H}{[0,\delta]}{\tr{X}{i\delta}}\subseteq \ivflow{H}{\delta}{K}{\epsilon}{\tr{X}{i\delta}}$.
But we assumed $\reach{H}{i\delta}{X_0}\subseteq \tr{X}{i\delta}$.
Therefore,
%
\begin{align*}
& \reach{H}{(i+1)\delta}{X_0}
= \reach{H}{\delta}{\reach{H}{i\delta}{X_0}} \subseteq \reach{H}{\delta}{\tr{X}{i\delta}}
\subseteq \ztset{\flow{H}{\delta}{K}{\epsilon}{\tr{X}{i\delta}}}\\
& = \ztset{\tr{X}{i\delta}}.\%~\text{by Step~\ref{step:1} in Algorithm~\ref{alg:main}}\\
%
& \forall t\in(i\delta,(i+1)\delta)~\reach{H}{t}{X_0}
= \reach{H}{t-i\delta}{\reach{H}{i\delta}{X_0}} \subseteq \reach{H}{[0,\delta]}{\tr{X}{i\delta}}
\subseteq \ztset{\ivflow{H}{\delta}{K}{\epsilon}{\tr{X}{i\delta}}}\\
& = \ztset{\tr{X}{t}}. \%~\text{by Step~\ref{step:2} Algorithm~\ref{alg:main}}
\end{align*}
%
This proves the claim made above.

Then it follows by induction that the theorem is true if
$X_0\subseteq \tr{X}{0}$.  We have $\tr{X}{0} = \iouopt{\eta}{X_0}$,
which is constructed by dividing $X_0$ by different optimized division
vectors and intersecting the unions and results in a sound
overapproximation of $X_0$.  Therefore, the theorem is true.
\end{proof}
%
To illustrate the difference in accuracy between intersection of
unions and only union method, we ran both techniques on the
Example~\ref{eg:ill}.  We took $K=20$ and $\epsilon = 10^{-12}$.  The
plot of interval bounds in Figure~\ref{fig:ill} clearly shows the the
IoU method is far more accurate on this example.  In the next section,
we evaluate our method on real world examples. with higher dimensions
and compare them with state-of-the-art methods, besides the union
method.
%
\begin{algorithm}\caption{IoU interval zonotope
flowpipe computation}\label{alg:main}
\KwData{$n$-dimensional system with $m$ inputs $H = \rb{f,U,\reals^n}$,
$T\in\reals_{\geq 0}$, $X_0\in\intervals^n$.}

\KwResult{Flowpipe $\trfn{X}:[0,T]\rightarrow \set{\text{Set of IoU interval
zonotopes and intervals}}|~\tr{X}{0}\supseteq X_0$.}

Choose $\eta,K\in\integers$, $\epsilon,\delta\in\double_{>0}$.

% assigning initial IoU
$j\gets 1$,~$k\gets 1$\;~\label{algstep:assign1}
\For{$q\in\dopt{q}$~~\% In parallel}{
\For{$b\in D_q$}{~~\% In parallel
$\rb{\iouopt{\eta}{X_0}}_{jk} \gets \lin{\zerozon{n}{l}}{\sqb{0}_{n\times n}}{b}$\;
$k\gets k+1$\; } $j\gets j+1$\; }
% done

$\tr{X}{0}\gets \iouopt{\eta}{X_0}$.~\label{algstep:assign2}

$i\gets 0.$

\While{$i\delta\leq N$}{
\For{ $j\in \rows{\iouopt{\eta}{X_0}}$ ~~(\% In parallel)}{

\For{$k\in\cols{\iouopt{\eta}{X_0}}$~~(\% In parallel)}{

$\rb{\tr{X}{(i+1)\delta}}_{jk} \gets {\flow{H}{\delta}{K}{\epsilon}{\rb{\tr{X}{i\delta}}_{jk}}}$.~\label{step:1}

}
}
$\tr{X}{(i+1)\delta} \gets \refine\rb{\tr{X}{(i+1)\delta}}$

$\forall t\in(i\delta,(i+1)\delta)~ {\tr{X}{t}}
= \ivflow{H}{\delta}{K}{\epsilon}{\rb{\tr{X}{i\delta}}}$.~\label{step:2}

$i \gets i+1$.
}
$\forall t\in[0,T]$, $\forall a\in\double^n$, $\flowproj{t}\rb{a}
= \sup\rb{a^T\overline{\tr{X}{t}}}$.
\end{algorithm}
%
\begin{figure}
{\center
  \includegraphics[scale = 0.5]{illImages/Ub.png}
  
  \includegraphics[scale = 0.41]{illImages/leg1.png}~
  \includegraphics[scale = 0.41]{illImages/leg2.png}~
  \includegraphics[scale = 0.41]{illImages/leg3.png}
  }
  \caption{Flowpipe bounds at different time points for
    Example~\ref{eg:ill}}
  \label{fig:ill}
\end{figure}
%


\section{Evaluation}
We evaluate our intersection of unions method on two real world
examples, i.e., a $7$-dimensional nonlinear model of autonomous car
from Lavaei et. al.~\cite{lavaei2020formal} and $12$-dimensional
nonlinear model of quadrotor from ARCH 2020
workshop~\cite{geretti2020arch}.  On these examples, besides the union
method based on the performance index~(\ref{eqn:pi}) proposed in
Althoff et. al.~\cite{althoff2008reachability}, we compare with
with polynomial zonotopes~\cite{althoff2013reachability} in CORA
tool\footnote{\url{https://tumcps.github.io/CORA/}} and Taylor
models~\cite{chen2012taylor} in
Flowstar\footnote{https://flowstar.org}.

For computation of symbolic derivatives, we use Sympy Python
software~\cite{10.7717/peerj-cs.103} and for interval arithmetic, we
use Boost C++ library~\cite{bronnimann2006design}.  We use
OpenMp\footnote{\url{https://www.openmp.org//}} to run in
parallel each iteration Step~\ref{step:1} of the for loops of
Algorithm~\ref{alg:main}.

\emph{Settings:}  We
ran our IoU method, union method and Flowstar Taylor model on an AWS
c5a.16xlarge cluster with 64 virtual cpus.  We ran CORA in MATLAB
2019a on a 1.4 GHz laptop with 4 GB ram, 1600 MHz DDR3.  The
simulation time step is $\delta = 0.005\si{\second}$ for car model and
$\delta = 0.01\si{\second}$ for quadrotor model.  The simulation time
horizon is $[0, 5\si{\second}]$.  The order of Taylor expansion of
Flowstar Taylor model is $9$ for car model and $7$ for quadrotor
model.  For the CORA polynomial zonotope, our IoU method and the union
method, the generator order is $l=400$ for car model and $l=200$ for
quadrotor model.  The order of \emph{dependent generators} of
polynomial zonotope is $2000$ for both models.  We took $K = 20$ and
$\epsilon = 10^{-12}$.  For the union method, the normalization error
$\rho$ for division~(Equation~\ref{eqn:pi}) is $\rho
= \taylor{f}{X_0}{u}$, i.e., the Taylor error before division.

%% \subsection{Illustrative example: IoU  vs union of interval zonotopes}
%% We consider the 3-dimensional nonlinear system in Example~\ref{eg:ill}
%% \is{Does it represent some real system, for example, a bicycle? It would have been nice then.}
%% to illustrate the difference in accuracy between our IoU flowpipe and
%% the union flowpipe based on the performance index
%% in~\cite{althoff2008reachability}(Equation~\ref{eqn:pi} above).  The
%% initial set has to be large enough to demonstrate the effect of linearization error, which is $X_0 = [-1,1]^3$.

%% \emph{Settings:}  We ran both the IoU and unions algorithm on a 1.4 GHz
%% laptop with 4 GB ram, 1600 MHz DDR3 with 4 virtual cpu cores.
%% \is{If you are usimg the same machine configuration for all the experiments, then the above line should move before Section 4.1}
%% We consider $3$ different number of divisions for comparision, i.e.,
%% $2^\eta:\eta = 2, 3,4$, or $4, 8$ and $16$ divisions,
%% respectively. The order of interval zonotope is $l = 200$ and the time
%% step size is $\delta = 0.01\si{\second}$.  Furthermore, $\epsilon = 10^{-12}$
%% and $K = 20$.  The normalization error $\rho$ for division in the union method is
%% the Taylor error before splitting $\taylor{f}{X_0}{U}$.

%% \emph{Results:}  The upper and lower bounds on flowpipes at uniformly spaced time
%% points for the $x$ and $y$ variables is
%% shown in Figure~\ref{fig:ill}. 
%% The figure clearly demonstrates that our IoU
%% flowpipe is far more accurate than the union flowpipe for $x/y$
%% coordinates.  The bounds for $\theta$ are almost similar for both
%% methods and all types of divisions because of very less linearization
%% error $(\le 10^{-6})$.  The computation times for our IoU method are $6\si{\second}$, $12\si{\second}$ and $24\si{\second}$, respectively, for $4$, $8$ and $16$ divisions
%% per union.  The computation times will reduce significantly by using
%% more cpu cores.  We use more cpu cores for the higher dimensional
%% examples that we evaluate latter.
%% \is{The parallel computation aspect needs more details. May be in the Algorithm section, you may say how to parallelize your algorithm.}
%% %
\begin{figure}
\includegraphics[scale = 0.41]{autocarImages/ubToolSteering.png}\hspace{-2.2em}
\includegraphics[scale = 0.41]{autocarImages/ubToolSlip.png}
\includegraphics[scale = 0.41]{autocarImages/ubToolYaw.png}\hspace{-2.2em}
\includegraphics[scale = 0.41]{autocarImages/ubToolYawRate.png}
\includegraphics[scale = 0.41]{autocarImages/ubToolx2.png}\hspace{-2.2em}
\includegraphics[scale = 0.41]{autocarImages/ubToolx1.png}

\includegraphics[scale = 0.41]{autocarImages/leg1.png}~
  \includegraphics[scale = 0.41]{autocarImages/leg2.png}~
  \includegraphics[scale = 0.41]{autocarImages/leg3.png}
  \caption{Flowpipe bounds at different time points for
    car model}\label{fig:flowcar}
\end{figure}
%
\subsection{$7$-dimensional model of autonomous car}
A time discretized model of autonomous car manoeuvre with $7$-dimensional
state space and $2$ inputs was presented in~\cite{lavaei2020formal}.
We adapted this model into a continuous time system and provided a
stabilizing feedback with one of the inputs.  Then we have the
following $7$-dimensional system having only one
uncontrolled input.
%
\begin{align*}
\dot{x_1}  = & x_4\cos\rb{x_5+x_7}\hspace{2em} \dot{x_2} =
x_4\sin\rb{x_5+x_7}\\
%
\dot{x_3}  = & -r(x_5+x_7+x_3) \hspace{2em} \dot{x_4} =
 u \hspace{2em} \dot{x_5} = x_6 & \\
 %
 \dot{x_6}  = & \frac{\mu
 m}{I_z(l_r+l_f)}(l_fC_{Sf}(gl_r-uh_{cg})x_3+
 (l_rC_{Sr}(gl_f+uh_{cg})\\& -l_fC_{Sf}(gl_r-uh_{cg}))x_7
 -(l_fl_fC_{Sf}(gl_r-uh_{cg}) + l_rl_rC_{Sr}(gl_f+uh_{cg}))\frac{x_6}{x_4})\\
%
\dot{x_7} 
= & \frac{\mu}{x_4(l_r+l_f)}(C_{Sf}(gl_r-uh_{cg})x_3+(C_{Sr}(gl_f+uh_{cg})-C_{Sf}(gl_r-uh_{cg}))x_7\\
&-(l_f*C_{Sf}(gl_r-uh_{cg}) + l_rC_{Sr}(gl_f+uh_{cg}))x_6/x_4)-x_6
\end{align*}
%
Above, $x = \rb{x_1,\ldots,x_7}^T$ represents the state vector and $u$
is the input, while rest are constant parameters.  The 2-D position of
car is $(x_1,x_2)$, steering angle is $x_3$, heading velocity is
$x_4$, yaw angle is $x_5$, yaw rate is $x_6$ and slip angle is $x_7$.
The parameter values in S.I. units is 
%
$ g = 9.81, m = 1093.3, \mu = 1.0489, l_f = 1.156, l_r = 1.422, h_{cg}
  = 0.574, I_z = 1791.6, C_{Sf} = 20.89, C_{Sr} = 20.89, r = 4$.  We
  consider the input set $u\in U = [-0.01,0.01]$ and the initial set
  $X_0 =
  [-1,1]\times[-0.5,0.5]\times[-0.5,0.5]\times[5,6]\times[-0.25,0.25]\times[-0.2,0.2]\times[-0.25,0.25]$.
  We compare our IoU method with the union
  method~(Equation~\ref{eqn:pi}) and also Taylor models in Flowstar
  with high expansion order and polynomial zonotopes in CORA with
  large number of generators and dependent generators.

%% \emph{Settings:}  The simulation time step is $\delta = 0.005\si{\second}$.  
%% For the Flowstar Taylor model, the order of Taylor expansion is $9$.  
%% For the CORA polynomial zonotope, the generator order is $400$ and the order of dependent generators is $2000$.  
%% The zonotope order of our IoU method and the union method is $l=400$.  
%% Also, $K = 20$ and $\epsilon = 10^{-12}$.  
%% For the only union method, the normalization error
%%   error $\rho$ for division is the Taylor error before division
%%   $\taylor{f}{X_0}{u}$.  We ran our IoU method and the union method on
%%   an AWS c5a.16xlarge cluster with 64 virtual cpus.  For Flowstar, we
%%   used AWS t2.large instance.  We ran CORA in MATLAB 2019a on the
%%   computer specified in the previous illustrative example.

\emph{Results:}  The upper and lower bounds of flowpipe for different
  methods is plotted over uniformly space time stamps in
  Figure~\ref{fig:flowcar}.  Flowstar could not complete the Taylor
  model flowpipe simulation beyond $1.15\si{\second}$ due to large
  time discretization error.  The bottom 3 lines in the figure
  correspond to our IoU flowpipe method, which show convergence while
  all other plots are diverging.  The figure clearly shows a high
  increase in accuracy by using intersection of unions compared to
  other methods.  The simulation times are given in
  Table~\ref{tab:comptimes}.  The preprocessing time before simulation
  for IoU method is $18\si{\second}$.
  %
 %
\begin{table}
\caption{Computation times}\label{tab:comptimes}
\begin{tabular}{|l|c|c|c|c|c|}
\hline
Example & IoU  & IoU  & IoU  &
Polynomial & Taylor\\
& $\eta = 5$ & $\eta = 4$ & $\eta = 3$ & Zonotope & Model \\
\hline
& & & & & {\color{red}Incomplete}\\
Car & 252 s & 128 s & 108 s & 4068 s& {\color{red}[0,1.15s]:~216s}\\
\hline
& & & & &\\
Quadrotor & 169 s & 318 s & 614 s & 2382 s &
{\color{red}[0,4.13s]:~583 s} \\
\hline
\end{tabular}
\end{table}
%
\begin{figure}
%\vspace{-1.05em}
\includegraphics[scale = 0.41]{quadrotorImages/ubToolHeight.png}\hspace{-2.2em}
\includegraphics[scale = 0.41]{quadrotorImages/lbToolHeight.png}

\includegraphics[scale = 0.41]{quadrotorImages/leg1.png}~
  \includegraphics[scale = 0.41]{quadrotorImages/leg2.png}~
  \includegraphics[scale = 0.41]{quadrotorImages/leg3.png}
\caption{Flowpipe bounds on height at different time points.\\
Note: Union method resulted in infinite bounds (not plotted).}\label{fig:flowquadrotor}
\end{figure}
% 
\subsection{$12$-dimensional model of quadrotor}
We consider the $12$-dimensional model of a quadrotor with three
inputs presented in the ARCH 2020 friendly
competition~\cite{geretti2020arch}, whose dynamics is given in the
appendix.  It has a $12$-dimensional state vector $x
= \rb{p_n,p_e,h,u,v,w,\phi,\theta,\psi,p,q,r}$, where $h$ is the
height of the quadrotor and a $3$-dimensional input vector $u
= \rb{u_1,u_2,u_2}$.  We take a larger initial set than the one given
in the competition~\cite{geretti2020arch} so that there is significant
linearization error in the flowpipe.  Our initial set is in S.I. units
is $X_0 = [-0.8,0.8]^6\times[-0.5,0.5]^2\times[0,0]\times
[-1,1]^2\times[0,0]$.  The input set is $\rb{u_1,u_2,u_3}\in
[-0.99,1.01]\times[-0.01,0.01]^2$.  The goal of controller is to
stabilize the veritical height of the quadrotor at $h = 1m$.

%% \emph{Settings:}  The simulation time step is $\delta = 0.01\si{\second}$.  
%% For Flowstar Taylor model, the order of Taylor expansion is $7$.  For
%%   the CORA polynomial zonotope, generator order is $200$ and order of
%%   dependent generators is $2000$.  The zonotope order of our IoU
%%   method and the union method is $l=200$.  Also, $K = 20$ and
%%   $\epsilon = 10^{-12}$.  For the only union method, we take the
%%   normalization error $\rho$ as the Taylor error before division
%%   $\taylor{f}{X_0}{u}$.  We ran our IoU method and the union method on
%%   an AWS c5a.16xlarge cluster with 64 virtual cpus.  For Flowstar, we
%%   used AWS t2.large instance.  We ran CORA in MATLAB 2019a on the
%%   computer specified in the previous illustrative example.  The
%%   simulation time horizon is $[0, 5 s]$.
  
%% \is{The setting is mostly the same as the previous examples. Could we create a subsection on experimental setup before the examples and put these details there?}

\emph{Results:}  Our IoU method for each of the $8$, $16$ and $32$ divisions per
  eigenvector direction showed much higher accuracy than other
  methods.  The union method based on division using
  Equation~\ref{eqn:pi} resulted in very high linearization error and
  consequently infinite bounds $[-1,1]^{12}$ after $0.5 s$ simulation
  time, where the normalization error is taken to be $\rho
  = \taylor{f}{X_0}{u}$.  We plotted the flowpipe bounds for the
  height in Figure~\ref{fig:flowquadrotor}.  The figure clearly shows
  high increase in accuracy by using intersection of unions compared
  to other methods.  The computation times are given in
  Table~\ref{tab:comptimes}.  The preprocessing time before simulation
  for our IoU method is $44\si{\second}$.
%








\bibliographystyle{plain}
\bibliography{ref}




\end{document}
