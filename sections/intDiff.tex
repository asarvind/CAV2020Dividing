\paragraph{Basic notation}:  The set of real numbers is $\reals$, rationals numbers is $\rationals$
and integers is $\integers$.  For $\Psi\subseteq\reals$, $a\in\reals$
and $\bowtie\in\set{\geq,\leq,>,<}$, we denote $\Psi_{\bowtie a}
= \set{ x\in\Psi \st x \bowtie a }$.  The supremum of a subset of real
numbers $\Psi\subseteq\reals$ is
$\sup\rb{\Psi}\in\reals\bigcup\set{\infty}$ and the infimum is
$\inf\rb{\Psi}\in\reals\bigcup\set{-\infty}$.  The absolute value of a
real number $x$ is $\abs{x} = \sup\set{x,-x}$.  For two reals $r$ and
$s$, the result of multiplying $r$ and $s$ is $rs$, the result of
adding of $r$ and $s$ is $r+s$, the result of subtracting $s$ from $r$
is $r-s$ and the result of dividing $r$ by $s$ if $s\neq 0$ is
$\frac{r}{s}$.  For non-negative integer $i$ and real number $x$, the
result of raising $x$ to the power $i$ is $x^i$.  The set of $n\times
m$ matrices containing elements from a set $\Psi$ is $\Psi^{n\times
m}$.  For simplicity, we denote $\Psi^{n\times 1}$ as $\Psi^n$.  So,
the set of real vectors is $\reals^n$. The $i^{th}$ row $j^{th}$
column element of a matrix $X$ is denoted $X_{ij}$.  Similarly, for
real vector $x$, $x_i$ denotes the $i^{th}$ component of $x$.  For
$X\in\reals^{n\times m}$ and $Y\in\reals^{m\times p}$, the matrix
product is $XY\in\reals^{n\times p}$ where $\forall
i\in\set{1,\ldots,n},\forall j\in\set{1,\ldots,p},\rb{XY}_{ij}
= \sum_{k=1}^{m}X_{ik}Y_{kj}$.  The transpose of a matrix $X$ is $X^T$
where $X^T_{ij} = X_{ji}$.  The euclidean norm of a real vector
$x\in\reals^n$ is $\norm{x} = \rb{\sum_{i=1}^n\abs{x_i}^2}^{1/2}$.
The infinity norm is $\norm{x}_\infty
= \sup\set{\abs{x_1},...,\abs{x_n}}$.  For $r\in\reals_{> 0}$ and
$x\in\reals^n$, we define the open box of width $r$ around $x$ as
$\ball{x}{r} = \set{ x\in\reals^n: \norm{x}_\infty < r }$.

\paragraph{Function expressions}:  For every $n\in\integers_{>0}$, we
define a set $\fe{n}$, called \emph{function expressions}, recursively
as follows.
%
\begin{enumerate}
\item $\forall i\in\set{1,\ldots,n}, \proj{i}\in\fe{n}$.
\item $\forall a\in\rationals, \forall x\in\fe{n}, af\in \fe{n}$.
\item $\forall f_1,f_2\in\fe{n}, \set{ \frac{1}{f}, f_1+f_2, f_1-f_2, f_1f_2}\subseteq\fe{n}$.
\item $\forall f\in\fe{n}, \set{ \sin\rb{f}, \cos\rb{f}, \exp\rb{f}, \log\rb{f} } \subseteq f$.
\end{enumerate}
%
Every function expression $f\in\fe{n}$ is associated with a \emph{partial
function}
$f:\reals^n\rightarrow\reals$ (i.e. domain is a subset of $\reals^n$)
defined recursively as follows.
\begin{enumerate}
\item If $f = \proj{i}$ for $i\in\set{1,\ldots,n}$, then $\forall
x\in\reals^n, f(x) = x_i$.
\item If $f = ag$ for $a\in\rationals, g\in\fe{n}$, then $\forall
x\in\reals^n, f\rb{x} = ag\rb{x}$.
\item If $f = hg$ for $h,g\in\fe{n}$, then $\forall x\in \reals^n, f\rb{x} =
h\rb{x}g\rb{x}$.
\item If $f = \frac{1}{g}$ for $g\in\fe{n}$, then $\forall
x \in\set{y\in\reals^n\st g\rb{y}\neq 0}$, $f\rb{x}
= \frac{1}{g\rb{x}}$.
\item If $f = f_1\bowtie f_2$ for $\bowtie\in\set{+,-}$ and
$f_1,f_2\in\fe{n}$, then $\forall x\in \reals^n, f\rb{x} = f_1\rb{x}\bowtie f_2\rb{x}$.
\item If $f = \sin\rb{g}$ for $g\in\fe{n}$, then $\forall x\in \reals^n, f\rb{x} = \sum_{N=0}^\infty(-1)^n\frac{\rb{g\rb{x}}^{2n+1}}{\rb{2n+1}!}$.
\item If $f = \cos\rb{g}$ for $g\in\fe{n}$, then $\forall x\in \reals^n, f\rb{x}
= \sum_{N=0}^\infty (-1)^n\frac{\rb{g\rb{x}}^{2n}}{\rb{2n}!}$.
\item If $f = \exp\rb{g}$ for $g\in\fe{n}$, then $\forall x\in \reals^n, f\rb{x}
= \sum_{N=0}^\infty(-1)^n\frac{\rb{g\rb{x}}^{n}}{\rb{n}!}$.
\item If $f = \log\rb{g}$ for $g\in\fe{n}$, then\\ $\forall
x \in\set{y\in\reals^n\st g\rb{y}\geq 0}, f\rb{x}
= \sum_{N=1}^\infty(-1)^{n+1}\frac{\rb{g\rb{x}-1}^{n}}{n}$.
\end{enumerate}
%
The domain of a function expression $f\in\fe{n}$, denoted $\dom{f}$, is the collection of
all possible $x\in\reals^n$ such that $f\rb{x}$ exists by the above
definition.
%
\begin{example}
Let us consider a real valued function
$\rb{x,y}\rightarrow \sin(x)\log(y) + \frac{\exp{y}}{x}$ which exists
only when $x\neq 0$ and $y\geq 0$.  The above function can be
represented by the function expression
$f = \sin\rb{\proj{1}}\log\rb{\proj{2}} + \frac{1}{\proj{1}}\exp\rb{\proj{2}}$
and $\dom{f} = \set{(x,y)\in\reals^2\st x\neq 0, y>0}$.
\end{example}
%
\paragraph{Symbolic Derivative}:  We say that the symbolic derivative of a function
expression $f\in\fe{n}$ is a tuple of function
expressions $\rb{g_1,\ldots,g_n}\in\fe{n}^n$ if all of the
following is true.
%
\begin{itemize}
\item $\forall i\in\set{1,\ldots,n}$, $\dom{f}\subseteq \dom{g}$.
\item For all $x\in\dom{f}$ and $\forall\epsilon\in\reals_{>0}$, there exists $r\in\reals_{>0}$ 
such that $\forall h\in\ball{0}{r}$, we have $(x+h)\in\dom{f}$ and
%
\begin{align*}
& \frac{\norm{f\rb{x+h}-\sum_{i=1}^ng_i\rb{x}h_i}}{\norm{h}}
< \epsilon.~\numberthis\label{eqn:symder}
\end{align*}
%
\end{itemize}
%
In the above case, we say the tuple $\rb{g_1,\ldots,g_n}$ is a
symbolic derivative of $f$.
\begin{example}
Let us consider a function expression
$f:\proj{1}\rb{sin\rb{\proj{2}}}$ which is associated with the
function $(x,y)\rightarrow x\rb{sin(y)}$.  By applying the basic rules
of calculus, we get
$\frac{\partial{f}}{\partial{x}}\left|_{x,y}\right. = sin(y)$ and
$\frac{\partial{f}}{\partial{y}}\left|_{x,y}\right. =
x\rb{cos(y)}$. So, the pair of function expressions
$\rb{ \sin\rb{\proj{2}}, \proj{1}\rb{cos\rb{\proj{2}}}}$ is a symbolic
derivative of $f$.
\end{example}
%
There are a number of softwares to compute symbolic derivatives for
the function expressions we defined above.  For example, in the
experiments in this paper,
we use the \emph{SymPy} software~\cite{10.7717/peerj-cs.103} to
compute symbolic derivatives.  We denote the symbolic derivative of
$f$ computed using a any such method as $\nabla f$ whose $i^{th}$
component is denoted $\nabla_i f$.
%
\paragraph{Interval arithmetic}:  We use interval arithmetic to
correctly bound floating point computations like addition,
multiplication, inverse and also bound values of sine, cosine and
logarithmic functions.  Let $\double$ represent the subset of real
numbers that can be represented in double precision floating point
format.  For $a,b\in\double$ such that $a<b$, we denote
$\interval{a}{b} = \set{x\in\reals\st a\leq x\leq b}$,
$\interval{-\infty}{a} = \set{x\in\reals\st x\leq a}$,
$\interval{a}{\infty} = \set{x\in\reals\st a\leq x}$ and
$\interval{-\infty}{\infty} = \reals$.  We then denote $\intervals
= \set{\interval{a}{b}\st a,b\in\double\bigcup\set{-\infty,\infty}}$.
An interval arithmetic describes computations over $\intervals$ where
the following property, called \emph{inclusion property}, is true.
For all $u,v\in\intervals$, all of the following is true.
%
\begin{enumerate}
\item For $a\in\rationals$, $au\in\rationals$ such that $au\subseteq \set{ax\st x\in\reals}$.
\item $uv,(u+v),(u-v)\in\intervals$ such that
$uv\subseteq\set{xy\st x\in u, y\in v}$,\\ $\rb{u+v}\subseteq\set{x+y\st x\in
u, y\in v}$ and $\rb{u-v}\subseteq\set{x-y\st x\in u, y\in v}$.
\item If $u\subseteq\reals\setminus\set{0}$, then $\frac{1}{u}\in\intervals$ such that $\frac{1}{u}\subseteq\set{\frac{1}{x}\st x\in u}$.
\item $\sin\rb{u}\in\intervals$ such that $\sin\rb{u}\subseteq\set{\sum_{N=0}^\infty(-1)^n\frac{\rb{x}^{2n+1}}{\rb{2n+1}!}\st x\in u}$.
\item $\cos\rb{u}\in\intervals$ such that $\cos\rb{u}\subseteq\set{\sum_{N=0}^\infty (-1)^n\frac{\rb{x}^{2n}}{\rb{2n}!}\st x\in u}$.
\item $\exp\rb{u}\in\intervals$ such that $\exp\rb{u}\subseteq\set{\sum_{N=0}^\infty(-1)^n\frac{\rb{x}^{n}}{\rb{n}!}\st x\in u}$.
\item If $u\subseteq\reals\setminus\set{0}$, then $\log\rb{u}\in\intervals$ such that\\ $\log\rb{u}\subseteq\set{\sum_{N=1}^\infty(-1)^{n+1}\frac{\rb{x-1}^{n}}{n}\st x\in u}$.
\end{enumerate}
%
Given intervals $u,v\in\intervals$, we use the \emph{Boost interval
arithmetic library}~\cite{bronnimann2006design} to compute the intervals
$uv, \rb{u+v}, \rb{u-v}, \rb{u}^{-1}, \sin\rb{u}, \cos\rb{u}, \exp\rb{u}$
and $\log\rb{u}$ such that the above inclusion property is true.

We denote the set of $n\times m$ interval matrices by $\intervals^{n\times m}$.
The set of interval vectors is $\intervals^{n\times 1}$, conveniently
denoted $\intervals^{n}$.  For $A\in\reals^{n\times m}$ and
$X\in\intervals^{n\times m}$, we say $A\in X$ if
$\forall \rb{i,j}\in\set{1,\ldots,n}^2, A_{ij}\subseteq X_{ij}$.  The
matrix product $XY$ of two interval matrices $X\in\intervals^{n\times
p}$ and $Y\in\intervals^{p\times m}$ is $XY
= \sum_{k=1}^pX_{ik}Y_{kj}$.  It follows from the inclusion property
of interval arithmetic that if $A\in X$ and $B\in Y$, then $AB\in XY$.
Given a function expression $f\in\fe{n}$ and an interval vector
$x\in\intervals^n$, based on interval arithmetic we compute interval
bounds $\iv{f}{x}$ on its image $f\rb{x}$, recursively as follows.
%
\begin{enumerate}
\item If $f = \proj{i}$ for $i\in\set{1,\ldots,n}$, then $\iv{f}{x} =
x_i$.
\item If $f = ag$ for $a\in\rationals, g\in\fe{n}$, then $\iv{f}{x} =
a\iv{f}{x}$.
\item If $f = hg$ for $h,g\in\fe{n}$, then $\iv{f}{x}
= \iv{h}{x}\iv{g}{x}$.
\item If $f = \frac{1}{g}$ for $g\in\fe{n}$ such that
$\iv{g}{x}\subseteq\reals\setminus \set{0}$, then $\iv{f}{x}
= \frac{1}{\iv{g}{x}}$.
\item If $f = f_1\bowtie f_2$ for $\bowtie\in\set{+,-}$ and
$f_1,f_2\in\fe{n}$, then $\iv{f}{x} = \iv{f_1}{x}\bowtie\iv{f_2}{x}$.
\item If $f = h\rb{g}$ for $h\in\set{\sin, \cos, \exp, \log}$, then
$\iv{f}{x} = h\rb{\iv{g}{x}}$.
\end{enumerate}
%


