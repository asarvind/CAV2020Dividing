An $n$-dimensional nonlinear dynamical system with $m$ inputs is
specified by a tuple $\rb{f,\inpset,\Omega}$, where $f =
\rb{f_1,\ldots,f_n}\in \fe{n+m}^n$ is an
$n$-tuple of partial functions in $\fe{n+m}^n$, an interval vector
$\inpset\in\intervals^m$ is called the \emph{input set}, and
$\Omega\in\intervals^n$ is called the \emph{state space}.  For any
$T\in\reals_{>0}$, a function \mbox{$\trfn{x}:[0,T)\rightarrow \reals^n$} is
called a \emph{state trajectory} if there exists a \emph{piecewise
continuous} function $\trfn{u}:[0,T)\rightarrow \inpset$,
called \emph{input trajectory}, such that all of the following is
true.
%
\begin{align*}
& \forall t\in[0,T), \tr{x}{t}\subseteq \Omega,\\
& \forall
t\in[0,T)\ \forall\epsilon\in\reals_{>0}, \exists\delta\in\reals_{>0}~\text{such
that}\\
& \ \ \ \  \forall h\in[-\delta,\delta]\bigcap[0,T]~\norm{ \frac{ \tr{x}{t+h}
- \tr{x}{t} }{h} - f_i\rb{ \mymatrix{ \tr{x}{t}\\ \tr{u}{t} }
}}< \epsilon~\numberthis\label{eqn:dynamics}
\end{align*}
%
In the rest of the paper, $n$ denotes the dimension of a nonlinear system
and $m$ denotes the number of inputs, unless otherwise stated.  For a
trajectory $\trfn{x}:[0,T)\rightarrow \reals^n$ and $\tau<T$, we
denote $\trfn{x}|_{[0,\tau]}
= \trfn{y}:[0,T)\rightarrow \reals^n: \forall t\in[0,\tau), \tr{y}{t}
= \tr{x}{t}$.
%
Our objective is to compute an over-approximation of the set of all state trajectories of a nonlinear system.
%
\begin{example}\label{eg:ill}
Let us denote the function expressions $\proj{1} = x$, $\proj{2} = y$,
$\proj{3} = \theta$ and $\proj{4} = u$.  A $3$-dimensional nonlinear
system with one input, state space $\reals^n$ and input set $U =
[-0.3,0.3]$ can be specified as $H = \rb{f,[-0.3,0.3],\reals^3}$ where
%
\begin{align*}
& f_1 = \rb{ u\rb{ x + \theta + 0.1y } - 1}x,
& f_2 = \rb{ u\rb{ y + \theta + 0.1x } - 1}y,\\
& f_3 = -\theta + 0.001\theta^2.
\end{align*}
%
Let us consider the zero input trajectory $\tr{u}{t} = 0~\forall
t\in[0,5)$.  In this case, two state trajectories are $\forall
t\in[0,5)$,
%
\begin{align*}
\tr{x}{t} = \mymatrix{ -0.1\exp\rb{-t}\\ 0.1\exp\rb{-t}\\ 0.1\exp\rb{-t} },~
& \tr{x}{t} = \mymatrix{ -0.2\exp\rb{-t}\\ -0.1\exp\rb{-t}\\ -0.1\exp\rb{-t} }
\end{align*}
%
which begin at $\tr{x}{0} = \mymatrix{-0.1,0.1,0.1}^T$ and $\tr{x}{0}
= \mymatrix{-0.2,-0.3,-0.1}^T$, respectively.
%
There are also state trajectories without an analytic form.

\is{Is it possible to provide some more insight of what exactly we would like to get as output?}
\end{example}
%
%% The \emph{reachable set} of the nonlinear system $H
%% =\rb{f,\inpset,\Omega}$ from an initial set $\init$ at a time point
%% $t$, denoted $\reach{H}{t}{\init}$, is a set containing all points
%% $\tr{x}{t}$ where $\trfn{x}$ is a state trajectory of $H$ and
%% $\tr{x}{0}\in\init$.  Similarly, $\reach{H}{[t_1,t_2]}{\init}
%% = \bigcup_{t\in[t_1,t_2]}\reach{H}{t}{\init}$.  Computing the exact
%% reachable set of nonlinear systems is computationally infeasible in
%% general because nonlinear ODEs may not have analytically closed form
%% solutions.  Instead, an overapproximation of the reachable set at each
%% time point, called \emph{flowpipe}, can be computed to verify
%% essential reachability properties like safety and stability.  The
%% flowpipe approximation of reachable set is
%% represented in a computer by a data structure which can be manipulated
%% efficiently to check reachability properties.  Such data structures
%% are called set representations, whose examples
%% are \emph{boxes}, \emph{polytopes}, \emph{zonotopes}, \emph{Taylor
%% models}, \emph{barrier certificates} and \emph{polynomial
%% zonotopes}~(\todo{cite all}).
%
\paragraph{Linearization:}  We say that a dynamical system
$L = \rb{g,U\times V,\Omega}$, where $U\in\intervals^m$ and
$V\in\intervals^n$ are both input sets and $g\in\fe{2n+m}^n$, is
linear if we have matrices $A\in\reals^{n\times n}$ and
$B\in\reals^{m\times n}$ such that for all $i\in\set{1,\ldots,n}$,
$x\in\reals^n$, $u\in U$ and $v\in V$, we have
%
\begin{align*}
& g_i\rb{\mymatrix{x^T & u^T & v^T}^T} = \rb{Ax + Bu + v}_i.
\end{align*}
%
For convenience, we also denote the above linear system as $L
= \rb{A,B,U,V,\Omega}$.  This linear system is said to overapproximate
a nonlinear system $H = \rb{f,\inpset,\Omega}$, denoted $L\succeq H$,
if for all $x\in\Omega$ and $u\in\inpset$, we have $f\rb{\mymatrix{x^T
& u^T}}^T - Ax - Bu \in V$.  In this case, the reachable set of $H$ is
a subset of the reachable set of $L$ at every time point as explained
in the following lemma.
%
\begin{lemma}\label{lem:inclin}
Let us consider that a linear system $L = \rb{A,B,U,V,\Omega}$ is an
overapproximation of a nonlinear system $H = \rb{f,U,\Omega}$.  Then
for all $t\in[0,\infty]$ and $\init\subseteq\Omega$, we have
$\reach{H}{t}{\init}\subseteq \reach{L}{t}{\init}$.
\end{lemma}
%
\begin{proof}
Let us consider a trajectory $\trfn{x}:[0,T)\rightarrow\Omega$ of the
nonlinear system $H$.  Then there exists an input trajectory
$\trfn{u}:[0,T)\rightarrow U$ of $H$ such that~(\ref{eqn:dynamics}) is
true.  Let us define $\forall t\in[0,T)~\tr{v}{t} =
f\rb{\mymatrix{\tr{x}{t}^T & \tr{u}{t}^T}}^T - A\tr{x}{t} -
B\tr{u}{t}$.  Since $L$ is an overapproximation of $H$, we have
$\forall t\in[0,T)$, $\tr{v}{t}\in V$.  Then by using
~(\ref{eqn:dynamics}, we have the input trajectory
$\mymatrix{\trfn{u} \\ \trfn{v}}:[0,T)\rightarrow U\times V$ where
$\forall t\in[0,T), \epsilon\in\reals_{>0}$, there exists
$\delta\in\reals_{>0}$ such that $\forall
h\in[-\delta,\delta]\bigcap\, [0,T)$,
%
\begin{align*}
& \norm{ \frac{ \tr{x}{t+h} - \tr{x}{t} }{h} -
f_i\rb{ \mymatrix{ \tr{x}{t}^T & \tr{u}{t}^T }^T }}\\
& = \norm{ \frac{ \tr{x}{t+h} - \tr{x}{t} }{h} -
\rb{A\tr{x}{t} + B\tr{u}{t} + \tr{v}{t}} } < \epsilon.
\end{align*}
%
Therefore, $\tr{x}:[0,T)\rightarrow \reals^n$ is also a state
trajectory of the linear system $L$.  So, it follows that
$\reach{H}{t}{\init}\subseteq \reach{L}{t}{\init}$ for all
$\init\subseteq \Omega$.
\end{proof}
%
There are efficient algorithms to compute flowpipes of linear systems
with good accuracy~\cite{girard2005reachability,girard2008efficient}.
So, it would be useful to overapproximate a nonlinear system by a
linear system while computing flowpipe, as follows.
%
\begin{lemma}[Linear overapproximation]\label{lem:linearization}
Let us consider a bounded interval vector $\Omega\in\intervals^n$ and
a nonlinear system $H = \rb{f,U,\Omega}$ where $f\in\fe{n+m}^n$ and
$U\in\intervals^m$ is a bounded interval vector.  Let us define $W
= \mymatrix{\Omega^T & U^T}^T$,
\begin{align*}
\begin{split}
\stmat{f}{\Omega}{U} =  \iv{{\nabla{f}_{1:n,1:n}}}{\mi{W} },
%
~\inpmat{f}{\Omega}{U}
= \iv{{\nabla{f}_{1:n,n+1:n+m}}}{ \mi{W} },
\end{split}\\
%
\begin{split}
& \rb{\taylor{f}{\Omega}{U}}_i
= \frac{1}{2}\rad{W}\rb{{\nabla^2f_i}\rb{W}}\rad{W}~\%
% = \frac{1}{2}\sum_{k=1}^{n+m}\sum_{j=1}^{n+m}\iv{\nabla_k\nabla_j{f_i}}{{W}}\rad{W_k}\rad{W_j},~\%
\text{Taylor err.}\\
%
& \gamma\rb{f,\Omega,U} = 2\rad{ \stmat{f}{\Omega}{U}\Omega +
\inpmat{f}{\Omega}{U}U
},~~~\%\text{Floating point error bounds}
\end{split}\\
%
\begin{split}
\err{f}{\Omega}{U} = \iv{f}{ \mi{W} }
+ \taylor{f}{\Omega}{U} + \gamma\rb{f,\Omega,U}
- \stmat{f}{\Omega}{U}\mi{\Omega} - \inpmat{f}{\Omega}{U}\mi{U}.
\end{split}~\numberthis\label{eqn:linearize}
\end{align*}
Then $
H \preceq
\rb{\inf\rb{\stmat{f}{\Omega}{U}},
\inf\rb{\inpmat{f}{\Omega}{U}},
U,
\err{f}{\Omega}{U}
} $
%
\end{lemma}
%
\begin{proof}
Based on Taylor remainder theorem for expansion around the center
$\mi{W}$, for all $i\in\set{1,\ldots,n}$ and $z\in W$, there exists
$r\in W$ such that $f_i(z) = $
%
\begin{align*}
& f_i\rb{\mi{W}} + \nabla f_i\rb{\mi{W}}\rb{z - \mi{W}}
+ \frac{1}{2}\rad{W}\rb{{\nabla^2f_i}\rb{W}}\rad{W}~\numberthis\label{eqn:taylor}
\end{align*}
%
The lemma follows by substituting both $r$ and $z$ by
the interval vector $W$ in the R.H.S of Equation~(\ref{eqn:taylor}) and applying
interval arithmetic.
\end{proof}
%
\begin{example}
Let us consider the nonlinear system $H = \rb{f,[-0.2,0.2],[-1,1]^3}$
where $f$ is the tuple specified in Example~\ref{eg:ill}.  By applying
Lemma~\ref{lem:linearization}, we get a linear overapproximation $L
= \rb{A,B,[-0.2,0.2],V,[-1,1]^3}\succeq H$ where we illustrate below
how $A_{11}$, $B_{11}$, and $V_{1}$ are computed.
%
\begin{align*}
& A_{11} = \inf\rb{\rb{\frac{df_1}{dx}}\mid_{\rb{x,y,z,u} =
\ivmat{(0,0,0,0)}}} \\ & = \inf\rb{\rb{u\rb{ x + \theta + 0.1y } - 1 +
xu}\mid_{\rb{x,y,z,u} = \ivmat{(0,0,0,0)}}} = -1 \\
%
& B_{11} = \inf\rb{\rb{\frac{df_1}{du}}\mid_{\rb{x,y,z,u} =
\ivmat{(0,0,0,0)}}} \\ &= \inf\rb{\rb{x\rb{ x + \theta + 0.1y
}}\mid_{\rb{x,y,z,u} = \ivmat{(0,0,0,0)}}} = 0 \\
%
& \nabla_1\nabla_2f_1\rb{W_1-0}\rb{W_2-0} =
0.1*[-0.2,0.2]*[-1,1]*[-1,1],~\text{likewise},\\
& \taylor{f}{[-1,1]^3}{[-0.2,0.2]}
= \sum_{k=1}^4\sum_{j=1}^4\nabla_k\nabla_jf_1\rb{W_j-0}\rb{W_k-0}
%% & = 1/2*(2*[-0.2,0.2]*[-1,1]*[-1,1] + 0.1*[-0.2,0.2]*[-1,1]*[-1,1] +\\
%% & [-0.2,0.2]*[-1,1]*[-1,1] + \\
%% &  \rb{[-1,1]+[-1,1]+[0.1,0.1]*[-1,1]}*[-0.2,0.2]*[-1,1] + \\ 
%% & 0.1*[-0.2,0.2]*[-1,1]*[-1,1] + 0 + 0 + 0 + \\ 
%% & [-0.2,0.2]*[-1,1]*[-1,1] + 0 + 0 + 0 + \\
%% & \rb{[-1,1]+[-1,1]+[0.1,0.1]*[-1,1]}*[-0.2,0.2]*[-1,1] + 0 + 0 + 0) =\\
= [-1.26, 1.26]\\
%
& V_1 = \rb{\iv{f}{ 0 }
+ \taylor{f}{[-1,1]^3}{[-0.2,0.2]}
- A*0 - B*0}_1 \\
& = 0 + [-1.26,1.26] + 0 + 0 = [-1.26, 1.26]
\end{align*}
%
\end{example}
%
\paragraph{Reach set approximation of linear system:}  The reachable
set of a linear system $L = \rb{A,B,U,V,\Omega}$ in a time interval
$[t_1,t_2]$ has the following overapproximation.
%
\begin{lemma}\label{lem:linreach}
Let us consider
\begin{align*}
& \dismat{A}{L}{[t_1,t_2]} = \ivmat{\identity{n}}
+ \ivmat{A}[t_1,t_2]
+ \ivmat{A}{[t_1,t_2]}{[t_1,t_2]}
+ \ivmat{A}[0,t_2][0,t_2][0,t_2]\\
& \dismat{B}{L}{[t_1,t_2]}
= \ivmat{B}{[t_1,t_2]}
+ \ivmat{B}\ivmat{A}[0,t_2][0,t_2]\\
& \dismat{C}{L}{[t_1,t_2]}
= \ivmat{\identity{n}}{[t_1,t_2]}
+ \ivmat{\identity{n}}\ivmat{A}{[0,t_2]}[0,t_2].\\
%
& \text{Then},~\reach{L}{[t_1,t_2]}{\init} \subseteq \set{\dismat{A}{L}{[t_1,t_2]}x
+ \dismat{B}{L}{[t_1,t_2]}u + \dismat{C}{L}{[t_1,t_2]}v\st x\in \init, u\in U, v\in
V}.~\numberthis\label{eqn:linreach}
\end{align*}
\end{lemma}
%
\begin{proof}
It is known~(see Section 3.1 of~\cite{girard2005reachability}) that
the reachable set of the linear system at time $t$ is given by
%
\begin{equation*}
\begin{split}
& \left\{\exp\rb{At}x + \int_{\tau=0}^t\exp\rb{A\rb{t-\tau}}B\tr{u}{\tau}
d\tau\right.\\ &\left.
+ \int_{\tau=0}^t\exp\rb{A\rb{t-\tau}}\tr{v}{\tau}
d\tau\st{\forall \tau\in[0,t], \tr{u}{\tau}\in U,\tr{v}{\tau}\in
v}\right\}.
\end{split}
\end{equation*}
%
By using Taylor remainder theorem and interval arithmetic, we get for all
$t\in[t_1,t_2]~\exp\rb{At}\in
\dismat{A}{L}{[t_1,t_2]}$, $\forall \tau\in[0,t]~\exp\rb{A\rb{t-\tau}}B\in\ivmat{B}+\ivmat{B}\ivmat{A}[0,t],$
$\exp\rb{A\rb{t-\tau}}\in\ivmat{\identity{n}}+\ivmat{A}[0,t]$.  By
substituting these bounds in the above integrals, we get the result.
\end{proof}
%
In our paper, we introduce a variant of
zonotope~\cite{girard2005reachability}, called \emph{interval
zonotope}, and use them to represent the overapproximation of the
resulting set in~(\ref{eqn:linreach}) on a computer.  Interval
zonotope is a generalization of zonotope to soundly overapproximate
floating point errors in set manipulation as well as better
approximate intersection between sets compared to zonotopes.  It is
defined as follows.
%
\begin{definition}[Interval Zonotope]
Let us consider $l\in\integers_{>0}$, $\gen\in\intervals^{n\times
ln},\cen\in\intervals^{n}$ and $\bounds\in\intervals^n$.  An interval
zonotope of order $l>1$ is the tuple $\rb{\gen,\cen,\bounds}$ which
represents the following set.
%
\begin{align*}
\iz{\gen}{\cen}{\bounds}
= \set{x\in\reals^n \st{\exists \zeta\in[-1,1]^{nl}:~x\in\bounds,~x \in \gen\ivmat{\zeta}+\cen} }
\end{align*}
%
\end{definition}
%
Approximating the reachable set of a linear system requires
approximating linear transformation of sets.  The matrix
multiplication of an interval zonotope and its Minkowski
sum~\cite{girard2005reachability} with an interval vector is
overapproximated as follows.
%
\begin{lemma}[Linear transformation]\label{lem:lintrans}
Let us consider $A\in\intervals^{n\times n}$, $w\in\intervals^n$ and an
interval zonotope $\rb{{\gen},{\cen},{\bounds}}$ of order $l>1$.  Let us
consider $\gen^\prime\in\reals^{n\times nl}$ where 
%
\begin{align*}
& \gen^\prime_{{:,n+1:ln}} = A\gen_{\rb{:,1:(l-1)n}}\\
& \gen^\prime_{{:,1:n}}
= \diag{\sup\rb{\rb{A\gen_{\rb{:,(l-1)n+1:ln}} }[-1,1]^n
+ \rad{w} }}\\
& z = \rb{A\gen}[-1,1]^n + A\cen + w.\\
& \text{Then},~\set{A^\prime x + y\st{x\in\iz{\gen}{\cen}{\bounds}, y\in\bounds,
A^\prime\in A}} \\
& \subseteq \iz{\gen^\prime}{A\cen+\mi{w}}{\meet{\rb{A\bounds + A\cen 
+ w}}{z}}~\numberthis\label{eqn:linMin}
\end{align*}
%
\end{lemma}
%
\begin{proof}
Let us consider $x\in \iz{\gen}{\cen}{\bounds}$ and $y\in w$.  So,
there exists $\zeta\in [-1,1]^n$ such that $x \in \gen\zeta + \cen$ and
also $x\in\bounds$.  Then,
%
\begin{align*}
& Ax + y \in A\gen\zeta + A\cen + y\\
& = A\gen_{{:,1:(l-1)n}}\zeta_{{:,1:(l-1)n}} +
A\gen_{:,{(l-1)n+1:ln}}\zeta_{{(l-1)n+1:ln}} + A\cen + y\\
& \subseteq A\gen_{{:,1:(l-1)n}}\zeta_{{:,1:(l-1)n}} 
 + A\gen_{:,{(l-1)n+1:ln}}[-1,1]^n + A\cen + w\\
& = \gen^\prime_{{:,n+1:ln}}\zeta_{{:,1:(l-1)n}} +
\rb{A\gen_{:,{(l-1)n+1:ln}}[-1,1]^n + \rad{w}} + A\cen +  \mi{w} 
\end{align*}
%
We have $A\gen_{:,{(l-1)n+1:ln}}[-1,1]^n + \rad{w} = \set{\gen^\prime_{:,1:n}\zeta^\prime\st{\zeta^\prime\in[-1,1]^n}}$
So, there exists $\zeta^\prime \in[-1,1]^n$ such that
%
\begin{align*}
& Ax + y =  \gen^\prime_{{:,n+1:ln}}\zeta_{{:,1:(l-1)n}}
+ \gen^\prime_{:,1:n}\zeta^\prime + A\cen + \mi{w}\\
& = \gen\mymatrix{\zeta_{{:,1:(l-1)n}}\\\zeta^\prime} + A\cen + \mi{w}~\numberthis~\label{eqn:pr1}
\end{align*}
%
Also, we get the following two bounds.
%
\begin{align*}
 & Ax + y \in A\gen[-1,1]^n + A\cen + w = z~\numberthis\label{eqn:pr2}\\
 & Ax + y \in A\bounds + A\cen + w~\numberthis\label{eqn:pr3}
\end{align*}
%
By~(\ref{eqn:pr1}),(\ref{eqn:pr2}) and (\ref{eqn:pr3}), we get that
$Ax+y\in\iz{\gen^\prime}{A\cen+\mi{w}}{\meet{\rb{A\bounds + A\cen + w}}{z}}$
\end{proof}
%
For convenience, if $\zt = \iz{\gen}{\cen}{\bounds}$, we denote the
R.H.S of~(\ref{eqn:linMin}) as $\lin{\zt}{A}{w}$.  The $n$-dimensional
interval zonotope of order $l$ which is equivalent to zero is denoted
$\zerozon{n}{l}$.

We can compute an interval overapproximation of the reachable set
$\reach{H}{[0,t]}{\init}$ in the interval $[0,t]$ as follows.
%
\begin{lemma}~\label{lem:bloat}
Let us consider a nonlinear system $H = \rb{f,U,\reals^n}$,
$\epsilon\in[0,\infty)$, an interval zonotope $\zt
= \rb{{\gen},{\cen},{\bounds^0}}$, and a sequence of interval vectors
$\rb{\bounds^i}_{i=0}^\infty$ where
%
\begin{align*}
& L_i = \rb{ \inf\rb{\stmat{f}{d^i}{U}},
\inf\rb{\inpmat{f}{d^i}{U}},U,\err{f}{\rb{\bounds^i+[-\epsilon,\epsilon]^n}}{U} }\\
& \bounds^{i+1}
= \join{\rb{\bounds^i+[-\epsilon,\epsilon]^n}}{\rb{\dismat{A}{L_i}{[0,t]}\bounds^0
+ \dismat{B}{L_i}{[0,t]}U + \dismat{C}{L_i}{[0,t]}V}}.~\numberthis\label{eqn:iter}
\end{align*}
%
For $K\in\integers_{>0}$, let $\ivflow{H}{t}{K}{\epsilon}{\zt}
= \begin{cases}\bounds^{K}
& \text{if}~\rb{\dismat{A}{L_i}{[0,t]}\bounds^0
+ \dismat{B}{L}{[0,t]}U + \dismat{C}{L}{[0,t]}V} \subseteq
\bounds^K\\ \reals^n & otherwise \end{cases}$.  Then,
$\ivflow{H}{t}{K}{\epsilon}{\zt}\supseteq\reach{H}{[0,t]}{\zt}$.
\end{lemma}
%
\begin{proof}
We prove this by contradiction.  Let us denote $\Omega
= \ivflow{H}{t}{K}{\epsilon}{\zt}$.  Let us assume that
$\Omega\nsupseteq \reach{H}{[0,t]}{\init}$.  So, $\Omega\neq\reals^n$
and $\Omega = \bounds^K$.  Then there exists a state trajectory
$\trfn{x}:[0,T)\rightarrow\reals^n$ and $\tau\in[0,t]$ such that $T>t$
and $\tr{x}{\tau}\notin \bounds^K$.  So, there exists $\tau_{min}$ in
$[0,T)$ such that $\tau_{min}
= \inf\set{t\in[0,T)\st{\tr{x}{\tau}\notin \bounds^K}}$.  By
using~(\ref{eqn:dynamics}), we can show there exists
$\delta\in[0,\infty)$ such that $\forall h\in[-\delta,\delta]$
$\tr{x}{\tau_{\min}+h}\in \ball{\tr{x}{\tau_{min}}}{\epsilon/2}$
($\epsilon$ is the positive real considered in the lemma statement).
Similarly, we can show there exists $r\in[0,\tau_{min})$ such that
$\tr{x}{r}\in \ball{\tr{x}{\tau_{min}}}{\epsilon/2}$.  Combining both
inequalities above, we get $\forall
h\in[-\delta,\delta],~\tr{x}{\tau_{\min}+h}\in \ball{\tr{x}{r}}{\epsilon}$.
As $r\in[0,\tau_{min})$, we get $\tr{x}{r}\in\bounds^K$.  So, for all
$h\in[-\delta,\delta)$, we have $\tr{x}{\tau_{\min}+h}\in \bounds^K +
[-\epsilon,\epsilon]^n$.  Also, by the definition of $\tau_{min}$, for
all $\tau\in[0,\tau_{min}-\delta/2)$, $\tr{x}{\tau}\in \bounds^K +
[-\epsilon,\epsilon]^n$. Combining last two inequalities, we get for
all $\tau\in[0,\tau_{min}+\delta)$ $\tr{x}{\tau}\in\bounds +
[-\epsilon,\epsilon]^n$.  Therefore,
$\trfn{x}|_{[0,\tau_{min}+\delta)}$ is a trajectory of
$\rb{f,U,\bounds^K+[-\epsilon,\epsilon]^n}$

By Lemma~\ref{lem:linearization}, $L_k$ is an overapproximation of
$\rb{f,U,\bounds^k+[-\epsilon,\epsilon]^n}$ .  As we showed that
$\trfn{x}|_{[0,\tau_{min}+\delta)}$ is a trajectory of
$\rb{f,U,\bounds^K+[-\epsilon,\epsilon]^n}$, using
Lemmas~\ref{lem:inclin} and~\ref{lem:linreach} and
Equation~\ref{eqn:iter}, we get that $\forall h\in [0,\delta)$,
$\tr{x}{\tau_{min}+h}\in \dismat{A}{L_K}{[0,t]}\bounds^0
+ \dismat{B}{L_K}{[0,t]}U
+ \dismat{C}{L_K}{[0,t]}V \subseteq \bounds^{k}$.  But by the
definition of $\tau_{min}$, $\exists h\in
[0,\delta),~\tr{x}{\tau_{min}+h}\notin \bounds^K$.  So, the assumption
that $\Omega\nsupseteq \reach{H}{[0,t]}{\init}$ is wrong.  Hence,
$\Omega\supseteq \reach{H}{[0,t]}{\init}$.
\end{proof}
%
To compute the reachable set at time $t$ from an interval zonotope
$\zt$, we do the following.  Let $\Omega
= \ivflow{H}{t+\epsilon}{K}{\epsilon}{\zt}$.  From above lemma, we
have $\Omega\supseteq\reach{H}{[0,t+\epsilon]}{\ztset{\zt}}$.  Then we
compute a linear overapproximation of $\rb{f,U,\Omega}$ as\\ $L
= \rb{\inf\rb{\stmat{f}{\Omega}{U}},
\inf\rb{\inpmat{f}{\Omega}{U}},
U,
\err{f}{\Omega}{U}
}$ based on Lemma~\ref{lem:linearization}.  Next we compute the linear transformation
%
\[
\flow{H}{t}{K}{\epsilon}{\zt}
= \lin{\zt}{ \dismat{A}{L}{[t,t]}}{\dismat{B}{L}{[t,t]}U
+ \dismat{C}{L}{[t,t]}V }
\]
%
The following lemma states that $\ztset{\flow{H}{t}{K}{\epsilon}{\zt}
}$ is an overapproximation of the reachable set
$\reach{H}{t}{\ztset{\zt}}$.
%
\begin{lemma}~\label{lem:reachnonlin}
Let $H = \rb{f,U,\reals^n}$ be a nonlinear system and $\zt$ an
interval zonotope.  Then
$\reach{H}{t}{\ztset{\zt}} \subseteq \ztset{\flow{H}{t}{K}{\epsilon}{\zt}}$.
\end{lemma}
%
\begin{proof}
Let $\trfn{x}:[0,T)\rightarrow \reals^n$ be a trajectory of $H$ such
that $\tr{x}{0}\in\ztset{\zt}$.  Let $\Omega
= \ztset{\ivflow{H}{t+\epsilon}{K}{\epsilon}{\zt}}$.  By
Lemma~\ref{lem:bloat}, we get $\Omega\supseteq\reach{H}{[0,t+\epsilon]}{\zt}$.  So,
$\trfn{y} = \trfn{x}|_{[0,(t+\epsilon)]}$ is a trajectory of $H^\prime
= \rb{f,U,\Omega}$.  By Lemma~\ref{lem:linearization}, we get that\\ $L
= \rb{\inf\rb{\stmat{f}{\Omega}{U}},
\inf\rb{\inpmat{f}{\Omega}{U}},
U,
\err{f}{\Omega}{U}
}$ is an overapproximation of $H^\prime$ .  By Lemma~\ref{lem:inclin},
we get
$\reach{L}{t}{\ztset{\zt}}\supseteq\reach{H^\prime}{t}{\Omega}$.  By
Lemma~\ref{lem:lintrans} and~\ref{lem:linreach}, we have
$\ztset{\flow{H}{t}{K}{\epsilon}{\zt}} \supseteq \reach{L}{t}{\ztset{\zt}}$.
So,
$\flow{H}{t}{K}{\epsilon}{\zt}\supseteq\reach{H^\prime}{t}{\Omega}$.
We showed that $\trfn{y}$ is a trajectory of $H^\prime$. Also,
$\tr{y}{t} =\tr{x}{t}$ as $t\in[0,t+\epsilon)$.  So, by the previous
inequality we get $\tr{y}{t}
= \tr{x}{t}\in\ztset{\flow{H}{t}{K}{\epsilon}{\zt}}$.  As this is true
for all trajectories $\trfn{x}$ of $H$, we have
$\reach{H}{t}{\ztset{\zt}}\subseteq\flow{H}{t}{K}{\epsilon}{\zt}$.
\end{proof}
%
