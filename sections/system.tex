An $n$-dimensional nonlinear dynamical system with $m$ inputs is
specified by a tuple $\rb{f,\inpset,\Omega}$, where $f =
\rb{f_1,\ldots,f_n}\in \fe{n+m}^n$ is an
$n$-tuple from the subset of partial functions $\fe{n+m}$, an
interval vector $\inpset\in\intervals^m$ is called the \emph{input
set}, and $\Omega\in\intervals^n$ is called the \emph{state space}.
For any $T\in\reals_{>0}$, a
function \mbox{$\trfn{x}:[0,T)\rightarrow \reals^n$} is called
a \emph{state trajectory} if there exists a \emph{piecewise
continuous} function $\trfn{u}:[0,T)\rightarrow \inpset$,
called \emph{input trajectory}, such that all of the following is
true.
%
\begin{align*}
& \forall t\in[0,T), \tr{x}{t}\subseteq \Omega,\\
& \forall
t\in[0,T)\ \forall\epsilon\in\reals_{>0}, \exists\delta\in\reals_{>0}~\text{such
that}\\
& \ \ \ \  \forall h\in[-\delta,\delta]\bigcap[0,T]~\norm{ \frac{ \tr{x}{t+h}
- \tr{x}{t} }{h} - f_i\rb{ \mymatrix{ \tr{x}{t}\\ \tr{u}{t} }
}}< \epsilon~\numberthis\label{eqn:dynamics}
\end{align*}
%
In the rest of the paper, $n$ is the dimension of a nonlinear system
and $m$ is the number of inputs, unless otherwise stated.  For a
trajectory $\trfn{x}:[0,T)\rightarrow \reals^n$ and $\tau<T$, we
define $\trfn{x}|_{[0,\tau]}
= \trfn{y}:[0,T)\rightarrow \reals^n: \forall t\in[0,\tau), \tr{y}{t}
= \tr{x}{t}$.   The \emph{reachable set} of the nonlinear system $H
=\rb{f,\inpset,\Omega}$ from an initial set $\init$ at a time point
$t$, denoted $\reach{H}{t}{\init}$, is a set containing all points
$\tr{x}{t}$ where $\trfn{x}$ is a state trajectory of $H$ and
$\tr{x}{0}\in\init$.  Similarly, $\reach{H}{[t_1,t_2]}{\init}
= \bigcup_{t\in[t_1,t_2]}\reach{H}{t}{\init}$.
%
\begin{example}\label{eg:ill}
Let us denote the function expressions $\proj{1} = x$, $\proj{2} = y$,
$\proj{3} = \theta$ and $\proj{4} = u$.  A $3$-dimensional nonlinear
system with one input, state space $\reals^n$ and input set $U =
[-0.3,0.3]$ can be specified as $H = \rb{f,[-0.3,0.3],\reals^3}$ where
%
\begin{align*}
& f_1 = \rb{ u\rb{ x + \theta + 0.1y } - 1}x,
& f_2 = \rb{ u\rb{ y + \theta + 0.1x } - 1}y,\\
& f_3 = -\theta + 0.001\theta^2.
\end{align*}
%
Let us consider the zero input trajectory $\tr{u}{t} = 0~\forall
t\in[0,5)$.  In this case, two state trajectories are $\forall
t\in[0,5)$,
%
\begin{align*}
\tr{x}{t} = \mymatrix{ -0.1\exp\rb{-t}\\ 0.1\exp\rb{-t}\\ 0.1\exp\rb{-t} },~
& \tr{x}{t} = \mymatrix{ -0.2\exp\rb{-t}\\ -0.1\exp\rb{-t}\\ -0.1\exp\rb{-t} }
\end{align*}
%
which begin at $\tr{x}{0} = \mymatrix{-0.1,0.1,0.1}^T$ and $\tr{x}{0}
= \mymatrix{-0.2,-0.3,-0.1}^T$, respectively.
%
But there are also state trajectories without an analytic form.  The
objective of flowpipe computation is to compute bounds along different
directions on the reachable set $\reach{H}{t}{\init}$ at each time
point in $[0,T)$, which originate $\init$ (eg. see Fig~\ref{fig:ill}).
\end{example}
%
%%  Computing the exact
%% reachable set of nonlinear systems is computationally infeasible in
%% general because nonlinear ODEs may not have analytically closed form
%% solutions.  Instead, an overapproximation of the reachable set at each
%% time point, called \emph{flowpipe}, can be computed to verify
%% essential reachability properties like safety and stability.  The
%% flowpipe approximation of reachable set is
%% represented in a computer by a data structure which can be manipulated
%% efficiently to check reachability properties.  Such data structures
%% are called set representations, whose examples
%% are \emph{boxes}, \emph{polytopes}, \emph{zonotopes}, \emph{Taylor
%% models}, \emph{barrier certificates} and \emph{polynomial
%% zonotopes}~(\todo{cite all}).
%
\paragraph{Linearization:}  We say that a dynamical system
$L = \rb{g,U\times V,\Omega}$, where $U\in\intervals^m$ and
$V\in\intervals^n$ are both input sets and $g\in\fe{2n+m}^n$, is
linear if we have matrices $A\in\reals^{n\times n}$ and
$B\in\reals^{m\times n}$ such that for all $i\in\set{1,\ldots,n}$,
$x\in\reals^n$, $u\in U$ and $v\in V$, we have $ g_i\rb{\mymatrix{x^T
& u^T & v^T}^T} = \rb{Ax + Bu + v}_i$.  For convenience, we also
denote the above linear system as $L = \rb{A,B,U,V,\Omega}$.  This
linear system is said to overapproximate a nonlinear system $H
= \rb{f,\inpset,\Omega}$, denoted $L\succeq H$, if for all
$x\in\Omega$ and $u\in\inpset$, we have $f\rb{\mymatrix{x^T & u^T}}^T
- Ax - Bu \in V$.  In this case, the reachable set of $H$ is a subset
of the reachable set of $L$ at every time point as explained in the
following lemma.
%
\begin{lemma}\label{lem:inclin}
Let us consider that a linear system $L = \rb{A,B,U,V,\Omega}$ is an
overapproximation of a nonlinear system $H = \rb{f,U,\Omega}$.  Then
for all $t\in[0,\infty]$ and $\init\subseteq\Omega$, we have
$\reach{H}{t}{\init}\subseteq \reach{L}{t}{\init}$.
\end{lemma}
%
\begin{proof}
See appendix.
\end{proof}
%
There are efficient algorithms to compute flowpipes of linear systems
with good accuracy~\cite{girard2005reachability,girard2008efficient}.
So, it would be useful to overapproximate a nonlinear system by a
linear system while computing flowpipe, as follows.
%
\begin{lemma}[Linear overapproximation]\label{lem:linearization}
Let us consider a bounded interval vector $\Omega\in\intervals^n$ and
a nonlinear system $H = \rb{f,U,\Omega}$ where $f\in\fe{n+m}^n$ and
$U\in\intervals^m$ is a bounded interval vector.  Let us define $W
= \mymatrix{\Omega^T & U^T}^T$,
\begin{align*}
\begin{split}
\stmat{f}{\Omega}{U} =  \iv{{\nabla{f}_{1:n,1:n}}}{\mi{W} },
%
~\inpmat{f}{\Omega}{U}
= \iv{{\nabla{f}_{1:n,n+1:n+m}}}{ \mi{W} },
\end{split}\\
%
\begin{split}
& \rb{\taylor{f}{\Omega}{U}}_i
= \frac{1}{2}\rad{W}\rb{{\nabla^2f_i}\rb{W}}\rad{W}~\%
% = \frac{1}{2}\sum_{k=1}^{n+m}\sum_{j=1}^{n+m}\iv{\nabla_k\nabla_j{f_i}}{{W}}\rad{W_k}\rad{W_j},~\%
\text{Taylor err.}\\
%
& \gamma\rb{f,\Omega,U} = 2\rad{ \stmat{f}{\Omega}{U}\Omega +
\inpmat{f}{\Omega}{U}U
},~~~\%\text{Floating point error bounds}
\end{split}\\
%
\begin{split}
\err{f}{\Omega}{U} = \iv{f}{ \mi{W} }
+ \taylor{f}{\Omega}{U} + \gamma\rb{f,\Omega,U}
- \stmat{f}{\Omega}{U}\mi{\Omega} - \inpmat{f}{\Omega}{U}\mi{U}.
\end{split}~\numberthis\label{eqn:linearize}
\end{align*}
Then $
H \preceq
\rb{\inf\rb{\stmat{f}{\Omega}{U}},
\inf\rb{\inpmat{f}{\Omega}{U}},
U,
\err{f}{\Omega}{U}
} $
%
\end{lemma}
%
\begin{proof}
Based on Taylor remainder theorem for expansion around the center
$\mi{W}$, for all $i\in\set{1,\ldots,n}$ and $z\in W$, there exists
$r\in W$ such that $f_i(z) = $
%
\begin{align*}
& f_i\rb{\mi{W}} + \nabla f_i\rb{\mi{W}}\rb{z - \mi{W}}
+ \frac{1}{2}\rad{W}\rb{{\nabla^2f_i}\rb{W}}\rad{W}~\numberthis\label{eqn:taylor}
\end{align*}
%
The lemma follows by substituting both $r$ and $z$ by
the interval vector $W$ in the R.H.S of Equation~(\ref{eqn:taylor}) and applying
interval arithmetic.
\end{proof}
%
\begin{example}
Let us consider the nonlinear system $H = \rb{f,[-0.2,0.2],[-1,1]^3}$
where $f$ is the tuple specified in Example~\ref{eg:ill}.  By applying
Lemma~\ref{lem:linearization}, we get a linear overapproximation $L
= \rb{A,B,[-0.2,0.2],V,[-1,1]^3}\succeq H$ where we illustrate below
how $A_{11}$, $B_{11}$, and $V_{1}$ are computed.
%
\begin{align*}
& A_{11} = \inf\rb{\rb{\frac{df_1}{dx}}\mid_{\rb{x,y,z,u} =
\ivmat{(0,0,0,0)}}} \\ & = \inf\rb{\rb{u\rb{ x + \theta + 0.1y } - 1 +
xu}\mid_{\rb{x,y,z,u} = \ivmat{(0,0,0,0)}}} = -1 \\
%
& B_{11} = \inf\rb{\rb{\frac{df_1}{du}}\mid_{\rb{x,y,z,u} =
\ivmat{(0,0,0,0)}}} \\ &= \inf\rb{\rb{x\rb{ x + \theta + 0.1y
}}\mid_{\rb{x,y,z,u} = \ivmat{(0,0,0,0)}}} = 0 \\
%
& \nabla_1\nabla_2f_1\rb{W_1-0}\rb{W_2-0} =
0.1*[-0.2,0.2]*[-1,1]*[-1,1],~\text{likewise},\\
& \taylor{f}{[-1,1]^3}{[-0.2,0.2]}
= \sum_{k=1}^4\sum_{j=1}^4\nabla_k\nabla_jf_1\rb{W_j-0}\rb{W_k-0}
%% & = 1/2*(2*[-0.2,0.2]*[-1,1]*[-1,1] + 0.1*[-0.2,0.2]*[-1,1]*[-1,1] +\\
%% & [-0.2,0.2]*[-1,1]*[-1,1] + \\
%% &  \rb{[-1,1]+[-1,1]+[0.1,0.1]*[-1,1]}*[-0.2,0.2]*[-1,1] + \\ 
%% & 0.1*[-0.2,0.2]*[-1,1]*[-1,1] + 0 + 0 + 0 + \\ 
%% & [-0.2,0.2]*[-1,1]*[-1,1] + 0 + 0 + 0 + \\
%% & \rb{[-1,1]+[-1,1]+[0.1,0.1]*[-1,1]}*[-0.2,0.2]*[-1,1] + 0 + 0 + 0) =\\
= [-1.26, 1.26]\\
%
& V_1 = \rb{\iv{f}{ 0 }
+ \taylor{f}{[-1,1]^3}{[-0.2,0.2]}
- A*0 - B*0}_1 \\
& = 0 + [-1.26,1.26] + 0 + 0 = [-1.26, 1.26]
\end{align*}
%
\end{example}
%
\paragraph{Reach set approximation of linear system:}  The reachable
set of a linear system $L = \rb{A,B,U,V,\Omega}$ in a time interval
$[t_1,t_2]$ has the following overapproximation.
%
\begin{lemma}\label{lem:linreach}
Let us consider
\begin{align*}
& \dismat{A}{L}{[t_1,t_2]} = \ivmat{\identity{n}}
+ \ivmat{A}[t_1,t_2]
+ \ivmat{A}{[t_1,t_2]}{[t_1,t_2]}
+ \ivmat{A}[0,t_2][0,t_2][0,t_2]\\
& \dismat{B}{L}{[t_1,t_2]}
= \ivmat{B}{[t_1,t_2]}
+ \ivmat{B}\ivmat{A}[0,t_2][0,t_2]\\
& \dismat{C}{L}{[t_1,t_2]}
= \ivmat{\identity{n}}{[t_1,t_2]}
+ \ivmat{\identity{n}}\ivmat{A}{[0,t_2]}[0,t_2].\\
%
& \text{Then},~\reach{L}{[t_1,t_2]}{\init} \subseteq \set{\dismat{A}{L}{[t_1,t_2]}x
+ \dismat{B}{L}{[t_1,t_2]}u + \dismat{C}{L}{[t_1,t_2]}v\st x\in \init, u\in U, v\in
V}.~\numberthis\label{eqn:linreach}
\end{align*}
\end{lemma}
%
\begin{proof}
See appendix.
\end{proof}
%
We can compute a crude interval overapproximation of the reachable set
$\reach{H}{[0,t]}{\init}$ in the interval $[0,t]$, as follows.  But we
can compute more accurate bounds by using advanced procedures which make
use of the below result as a subprocedure.
%
\begin{lemma}~\label{lem:bloat}
Let us consider a nonlinear system $H = \rb{f,U,\reals^n}$,
$\epsilon\in[0,\infty)$, a bounded interval vector
$\bounds^{0}\in\intervals^n$, and a sequence of interval vectors
$\rb{\bounds^i}_{i=0}^\infty$ where
%
\begin{align*}
& L_i = \rb{ \inf\rb{\stmat{f}{d^i}{U}},
\inf\rb{\inpmat{f}{d^i}{U}},U,\err{f}{\rb{\bounds^i+[-\epsilon,\epsilon]^n}}{U} }\\
& \bounds^{i+1}
= \join{\rb{\bounds^i+[-\epsilon,\epsilon]^n}}{\rb{\dismat{A}{L_i}{[0,t]}\bounds^0
+ \dismat{B}{L_i}{[0,t]}U + \dismat{C}{L_i}{[0,t]}V}}.~\numberthis\label{eqn:iter}
\end{align*}
%
For $K\in\integers_{>0}$, let $\ivflow{H}{t}{K}{\epsilon}{\bounds^0}
= \begin{cases}\bounds^{K}
& \text{if}~\rb{\dismat{A}{L_i}{[0,t]}\bounds^0
+ \dismat{B}{L}{[0,t]}U + \dismat{C}{L}{[0,t]}V} \subseteq
\bounds^K\\ \reals^n & otherwise \end{cases}$.  Then,
$\ivflow{H}{t}{K}{\epsilon}{\bounds^0}\supseteq\reach{H}{[0,t]}{\bounds^0}$.
\end{lemma}
%
%Proof of above lemma is in the appendix.
\begin{proof}
See appendix.
\end{proof}
