We consider an $n$-dimensional nonlinear system with $m$-inputs
$H=\rb{f,U,\reals^n}$.  A flowpipe of $H$ in a time interval $[0,T)$
from an initial set $\init\in\intervals^n$ is a mapping
$\trfn{X}:[0,T]\rightarrow \set{\Omega\st{\Omega\subseteq \reals^n}}$
such that $\forall t\in[0,T]~\reach{H}{t}{Q}\subseteq \tr{X}{t}$.  In
the previous section, we showed how to compute an overapproximation of
the reachable set at time $t$ starting from an interval zonotope.
Using that, we can compute a flowpipe inductively as follows.  We
divide $[0,T]$ into small time intervals
$\rb{[0,\delta],[\delta,2\delta],\ldots,[N\delta,T]}$.  Then,
%
\begin{enumerate}
\item $\tr{X}{0} = \ztset{\rb{\diag{\rad{X_o}},\mi{X_0},X_0}}$,
i.e. convert initial set to an interval zonotope representation.
\item $\forall i\in\integers_{\geq 0}\tr{X}{(i+1)\delta} = \ztset{\flow{H}{t}{K}{\epsilon}{\tr{X}{i\delta}}}$.
\item $\forall t\in[i\delta,(i+1)\delta]\tr{X}{t} = \ztset{\ivflow{H}{t}{K}{\epsilon}{\tr{X}{i\delta}}}$.
\end{enumerate}
%
However, the above method incurs a Taylor error
$\taylor{f}{\ivflow{H}{t}{K}{\epsilon}{\tr{X}{i\delta}}}{U}$ at each
step in the above flowpipe computation.  If $\tr{X}{i\delta}
= \iz{\gen_i}{\cen_i}{\bounds_i}$, the Taylor error is positively
correlated to the the width $\sup\rb{\rad{\bounds_i}}$ of the interval
bounds of the interval zonotope~(see Equations~\ref{eqn:linearize}).
So, by dividing the reachable set $\tr{X}{i\delta}$ and computing the
next reachable set as union of reachable sets of constituting pieces,
we can reduce the linearization error.  But dividing a set into sets
of a given fixed size results in exponential number of divisions in
the dimension of the state space, which can be computationally
prohibitive.  Instead, we can divide the reach set into a fixed number
of subsets.  For a fixed number of divisions $2^\eta\in\integers$,
there are many ways of dividing the reachable set.  So, we have to
find a division that minimizes the Taylor error.  In this context, a
performance index was proposed in~(\todo{cite}) to divide reachable
sets as follows.  A reachable set $\Omega$ is divided into two sets
$\Omega_1$ and $\Omega_2$ which minimizes the following quantity for a
user chosen threshold $\rho\in\reals^n_{>0}$.
%
\begin{align*}
\sup_{i=1}^n\rb{\sup{\taylor{f}{\Omega_1}{U}}}_i/\rho_i\sup_{i=1}^n\rb{\sup{\taylor{f}{\Omega_2}{U}}}_i/\rho_i~\numberthis\label{eqn:pi}
\end{align*}
%
\paragraph{Drawback:}  The above performance index takes into account a measure of the
overall linearization error along different coordinates.  But
minimizing this performance index need not significantly reduce the
linearization error along a specific direction.  For illustration, let
us consider the nonlinear system in Example~\ref{eg:ill}.  Let $\Omega
= [-1,1]\times[-0.5,0.5]\times[-0.1,0.1]$ be the reachable set at a
time point and $\rho = \taylor{f}{\Omega}{U}$ is the threshold for
splitting.  Using the above performance index, we get that $\Omega$
should be split along $x$-axis into $\Omega_1 =
[-1,0]\times[-0.5,0.5]\times[-0.1,0.1]$ and $\Omega_2 =
[0,1]\times[-0.5,0.5]\times[-0.1,0.1]$.  However, the linearization
error along $y$-axis $\rb{\taylor{f}{\Omega}{U}}_3$ least depends on
the width along $x$-axis due to the multiplication coefficient $0.1$
of $xy$ in $f_3$.  Instead, the best division to minimize the
$y$-directional linearization error is to divide along $y$-coordinate
as $\Omega = [-1,1]\times[-0.5,0]\times[-0.1,0.1]$ and $\Omega_2 =
[-1,1]\times[0,0.5]\times[-0.1,0.1]$.  This means reducing the
linearization error using the above performance index need not
significantly reduce the linearization error along every direction is
a given set of directions.  Besides, it is not known how to choose a
threshold $\rho$ for good accuracy of reach set approximation.

\paragraph{New method:}  In this paper, we propose an better approach to
reduce the linearization error along each individual direction in a
finite set of directions.  To do so, we use \emph{intersection of
unions} (IoU).  For each direction, we find an optimized division of
the reach set for reducing the linearization error.  Then we intersect
the union sets corresponding to the optimized division for
linearization error along each direction.  This way, we ensure that
the linearization error is significantly reduced along each direction
in the given set of directions.  As a heuristic, we choose the
directions as the eigenvectors of the nonlinear system at the origin.

We only divide the initial set to reduce the linearization error
because of two reasons.
%
\begin{enumerate}
\item Dividing a reach set increases the representation complexity due to
increase in the number of elements in the union.  So, iterative
division of reach set at various time stamps will blow up the
representation complexity.
\item  An interval zonotope which is not an interval can not
be divided accurately and reduces the accuracy of flowpipe.
\end{enumerate}
%
Therefore, we only divide the initial set, which is an interval, to
compute an intersection of unions representation.
%
\subsection{Computing set of optimized division vectors:}
Let us denote $\bounds=X_0$.  Given a vector of positive integer
values $q\in\integers_{>0}$, called \emph{division vector}, we define
a union of interval vectors which is a close overapproximation of $h$
with only floating point approximation error, as follows.  The set of
the set of divided intervals corresponding to $q$ is
%
\begin{align*}
D_q
= \left\{r_1\times...\times r_n\right.\\
 \left.\st{\forall i\in\set{1,\ldots,n}~r_i
= \ivmat{\inf\rb{\bounds_i}}+k_i\frac{\rad{\bounds}_i}{q_i}, k_i\in\set{1,\ldots,q_i}}\right\}.
\end{align*}
%
For all intervals in $D_q$, an upper bound on the Taylor error is
given below.
%
\begin{align*}
\rb{\utaylor{f}{U}{\bounds}{q}}_i
= \frac{1}{2}\sum_{k=1}^{n}\sum_{j=1}^{n}\rb{\nabla_k\nabla_j{f_i}\rb{h}}\frac{\rad{h_k}}{q_k}\frac{\rad{h_j}}{q_j}
\\
+ \frac{1}{2}\sum_{k=n+1}^{n+m}\sum_{j=n+1}^{n+m}\rb{\nabla_k\nabla_j{f_i}\rb{h}}\rad{U_k}\rad{U_j}
\end{align*}
%
For any $k\in\set{1,\ldots,n}$, the projection of the upper bound
along the $k^{th}$ eigenvector is given by
$\utaylor{f}{U}{\bounds}{q}\eig_{:,k}
= \utaylor{f}{U}{\bounds}{q}\real{\eig_{:,k}}
+ \iota\utaylor{f}{U}{\bounds}{q}\imag{\eig_{:,i}}$.  We select $q$ to
reduce the maximum absolute value of this bound, by using the
following greedy optimization.  Let us denote for any
$i\in\set{1,\ldots,n}$, ${q}^i\in\integers$ where ${q}^i_j
= \begin{cases}2q_j & j=i\\ q_j & i\neq j \end{cases}$.

\begin{algorithm}[H]
\caption{Optimizing division vector for $k^{th}$
eigenvector} $q\gets \mymatrix{1,\ldots,1}^T\in\integers^n$\;
\While{$\prod_{j=1}^nq_j<2^\eta$}{
$ind \gets \argmin_{i=1}^n\abs{\sup\rb{\utaylor{f}{U}{\bounds}{{q}^i}\eig_{:,k}}}$\;
$q\gets {q}^{ind}$
}
\end{algorithm}
%
We denote the set of optimized division vectors correponding to
minimizing the projection of Taylor error along different eigenvectors
as $\dopt{\eta}$, where $2^\eta$ is the maximum number of divisions.
%
\begin{example}
Let us consider the nonlinear system $H$ in Example~\ref{eg:ill}, an
interval vector $h = [0,1]^3$ and maximum number of divisions $\eta =
4$.  The eigenvectors at the origin are the coordinate vectors
$\mymatrix{1 & 0 & 0}^T$, $\mymatrix{0 & 1 & 0}^T$ and $\mymatrix{0 &
0 & 1}^T$.  To reduce the linearization error along $\mymatrix{1 & 0 &
0}^T$, the best possible division vector is $\mymatrix{2 & 1 & 2}^T$,
i.e., divide $x$ and $\theta$ coordinates.  For the direction
$\mymatrix{0 & 1 &0}^T$, the optimum division vector is $\mymatrix{1 &
2 & 2}^T$ and for the direction $\mymatrix{0 & 0 & 1}^T$, the optimum
division vector is $\mymatrix{1 & 1 & 4}^T$.  So, $\dopt{4}
= \set{\mymatrix{2 & 1 & 2}^T, \mymatrix{1 & 2 & 2}^T, \mymatrix{1 & 1
& 4}^T}$.  The three different division vectors results three
different kinds of divisions of $[0,1]^3$ whose intersection is
represented as an IoU of interval zonotopes.
\end{example}
%
\subsection{Casting initial set as intersection of unions}  
For a division vector $q\in\dopt{\eta}$, we can closely
overapproximate $X_0$ as a union of interval zonotopes $\bigcup_{b\in
D_q}\iz{\diag{\rad{b}}}{\mi{b}}{b}$ with only floating point
overapproximation error.  To reduce the Taylor error along each
eigenvector, we store the union corresponding to each optimized
division vector in $\dopt{\eta}$ and take their intersection.  The
resulting intersection of unions is
%
\begin{align*}
\bigcap_{q\in\dopt{\eta}}\bigcup_{b\in
D_q}{\iz{\diag{\rad{b}}}{\mi{b}}{b}}~\numberthis\label{eqn:iouopt}
\end{align*}
%
In a computer, we represent an intersection of unions of interval
zonotope sets as above with an matrix of interval zonotope tuples,
called IoU interval zonotope.
%
\begin{definition}[IoU interval zonotope]
An IoU zonotope $\zt$ is a matrix of interval zonotopes which
represents the set
%
\[
\ztset{\zt} = \bigcap_{i=1}^{\rows{\zt}}\bigcup_{i=1}^{\cols{\zt}}\ztset{\zt_{ij}}.
\]
%
\end{definition}
%
We represent the optimized intersection of unions in
Equation~\ref{eqn:iouopt} by an IoU interval zonotope, which we denote
$\iouopt{\eta}{h}$.
%
\subsection{Computing IoU flowpipe}
For an IoU interval zonotope, we can compute the reachable
set at time $t$ according to the following lemma.
%
\begin{lemma}~\label{lem:ioureach}
Let us consider an IoU of interval zonotopes $\zt$,
$K\in\integers_{\geq 0}$, $\epsilon\in\reals_{>0}$.  Let us consider
IoU interval zonotopes $\flow{H}{t}{K}{\epsilon}{\zt}$ and
$\ivflow{H}{t}{K}{\epsilon}{\zt}$ defined as
%
\begin{align*}
\rb{\flow{H}{t}{K}{\epsilon}{\zt}}_{ij}
= \flow{H}{t}{K}{\epsilon}{\zt_{ij}},~~~
\rb{\ivflow{H}{t}{K}{\epsilon}{\zt}}_{ij}
= \ivflow{H}{t}{K}{\epsilon}{\zt_{ij}}.
\end{align*}
b%
Then
$\reach{H}{t}{\zt}\subseteq \ztset{\flow{H}{t}{K}{\epsilon}{\zt}}$ and
$\reach{H}{[0,t]}{\zt}\subseteq \ivflow{H}{t}{K}{\epsilon}{\zt}$.
\end{lemma}
%
\begin{proof}
Let us consider a trajectory $\trfn{x}$ such that $\tr{x}{0}\in\ztset{\zt}$.
This means $\forall i\in\set{1,\ldots,N} \exists
j_i:\tr{x}{0}\in\ztset{\zt_{ij_i}}$.  By Lemma~\ref{lem:reachnonlin}, we get
that $\forall
i\in\set{1,\ldots,N}~\tr{x}{t}\in\ztset{\flow{H}{t}{K}{\epsilon}{\ztset{\zt_{ij_i}}}}$
and by Lemma~\ref{lem:bloat} we get that $\forall t\in[0,t]\forall
i\in\set{1,\ldots,N}~\tr{x}{t}\in\ivflow{H}{t}{K}{\epsilon}{\ztset{\zt_{ij_i}}}$.
This proves the lemma.
\end{proof}
%
For an IoU interval zonotope, we can refine the interval bounds in
each of its constituting elements as follows. 
%
\begin{lemma}~\label{lem:refine}
Let $\zt$ be an IoU interval zonotope where $\zt_{ij}
= \rb{\gen^{ij},\cen^{ij},h^{ij}}$.  Let us consider $\widehat{h}
= \bigwedge_{i=1}^{\rows{\zt}}\bigvee_{j=1}^{\cols{\zt}}h_{ij}$.  We
define a refined IoU interval zonotope $\refine\rb{\zt}$ as
%
\[
\rb{\refine\rb{\zt}}_{ij}
= \rb{\gen^{ij},{\cen^{ij}},{\meet{\bounds^{ij}}
{\widehat{h}}}}.
\]
%
Then $\ztset{\zt} = \ztset{\refine\rb{\zt}}$.
\end{lemma}
%
The above lemma is straighforward to derive.  In this refined
representation, we reduce the interval bounds in each constitution
element of the IoU based on the overall interval bounds $\widehat{h}$
given above.

Now we have the results to compute an optimized IoU zonotope flowpipe.
The flowpipe computation is described in Algorithm~\ref{alg:main}.
The correctness of the algorithm is proved in Theorem~\ref{thm:main}.
%
\begin{algorithm}~\label{alg:main}
\KwData{Nonlinear system with $m$ inputs $H = \rb{f,U,\reals^n}$,
$T\in\reals_{\geq 0}$, $X_0\in\intervals^n$.}

\KwResult{Flowpipe $\trfn{X}:[0,T]\rightarrow \set{\text{Set of IoU interval
zonotopes and intervals}}|~\tr{X}{0}\supseteq X_0$.}

Choose $\eta,K\in\integers$, $\epsilon,\delta\in\double_{>0}$.

$\tr{X}{0}\gets \iouopt{\eta}{X_0}$.

$i\gets 0.$

\While{$i\delta\leq N$}{
$\tr{X}{(i+1)\delta} = \refine\rb{\flow{H}{\delta}{K}{\epsilon}{\tr{X}{i\delta}}}$.~\label{step:1}

$\forall t\in(i\delta,(i+1)\delta) \tr{X}{t} = \ivflow{H}{\delta}{K}{\epsilon}{\tr{X}{i\delta}}$.~\label{step:2}

$i \gets i+1$.
}
\end{algorithm}
%
\begin{theorem}~\label{thm:main}
Let us consider $\trfn{X}$ computed in Algorithm~\ref{alg:main}.  For
all $t\in[0,T]$, $\reach{H}{t}{X_0}\subseteq \ztset{\tr{X}{t}}$.
\end{theorem}
%
\begin{proof}
We prove the theorem inductively.

\paragraph{Claim:}  Let us consider that for some
$i\in\integers_{\geq 0}:i\delta<T$, we have
$\reach{H}{i\delta}{X_0}\subseteq\tr{X}{i\delta}$.  Then $\forall
t\in[i\delta,(i+1)\delta]~\reach{H}{t}{X_0}\subseteq \ztset{\tr{X}{t}}$.

\paragraph{Proof of claim:}
By Lemmas~\ref{lem:ioureach} and~\ref{lem:refine}, we get
$\reach{H}{\delta}{\tr{X}{i\delta}}\subseteq \ztset{\flow{H}{\delta}{K}{\epsilon}{\tr{X}{i\delta}}}$
and\\
$\reach{H}{[0,\delta]}{\tr{X}{i\delta}}\subseteq \ivflow{H}{\delta}{K}{\epsilon}{\tr{X}{i\delta}}$.
But we assumed $\reach{H}{i\delta}{X_0}\subseteq \tr{X}{i\delta}$.
Therefore, we get
%
\begin{align*}
& \reach{H}{(i+1)\delta}{X_0}
= \reach{H}{\delta}{\reach{H}{i\delta}{X_0}} \subseteq \reach{H}{\delta}{\tr{X}{i\delta}}
\subseteq \ztset{\flow{H}{\delta}{K}{\epsilon}{\tr{X}{i\delta}}}\\
& = \ztset{\tr{X}{i\delta}}.\%~\text{by Step~\ref{step:1} in Algorithm~\ref{alg:main}}\\
%
& \forall t\in(i\delta,(i+1)\delta)~\reach{H}{t}{X_0}
= \reach{H}{t-i\delta}{\reach{H}{i\delta}{X_0}} \subseteq \reach{H}{[0,\delta]}{\tr{X}{i\delta}}
\subseteq \ztset{\ivflow{H}{\delta}{K}{\epsilon}{\tr{X}{i\delta}}}\\
& = \ztset{\tr{X}{t}}. \%~\text{by Step~\ref{step:2} Algorithm~\ref{alg:main}}
\end{align*}
%
This proves the claim made above.

Then it follows by induction that the theorem is true if
$X_0\subseteq \tr{X}{0}$.  We have $\tr{X}{0} = \iouopt{\eta}{X_0}$,
which is constructed by dividing $X_0$ by different optimized division
vectors and intersecting the unions and results in a sound
overapproximation, as explained previously.  Therefore, the theorem is
true.
\end{proof}
%
