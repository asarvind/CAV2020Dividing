The dynamics of many real world systems is modeled by ordinary
differential equations (ODEs).  These models typically contain
uncertainties in the parameters, inputs and initial conditions to
account for the unpredictable behavior of the system.  To ensure
reliable behavior of the systems, especially those which are safety
critical, these ODE models have to be verified to satisfy the
requirements for the correct operation of the system.  Verifying
properties of the ODE models (eg. safety, stability) requires
approximating the set of trajectories (solutions) of the ODE in the
presence of uncertainties.  But exactly computing the set of
trajectories of nonlinear ODEs can be computationally intractable,
because computing accurate images of nonlinear maps is at least NP
Hard~\cite{murty1985some}.  Alternatively, significant research has
been carried out on overapproximating the set of trajectories of
nonlinear uncertain ODEs by data
structures~\cite{chen2012taylor,testylier2013nltoolbox,althoff2013reachability,kochdumper2020sparse,ramdani2009hybrid,han2006reachability,maidens2014reachability},
referred to as \emph{flowpipes}~\cite{chen2012taylor}.

Nevertheless, computing accurate images of nonlinear multivariable
functions is at least NP Hard~\cite{murty1985some}.  Therefore, the
computational complexity of any flowpipe computation algorithm for
nonlinear systems can blow up beyond tractable limits while ensuring a
desired accuracy in high dimensions.  For instance, a commonly used
method in flowpipe computation is piecewise
linearization~\cite{ramdani2009hybrid,han2006reachability,althoff2008reachability,li2020reachability,dang2010accurate,bak2016scalable},
where the nonlinear ODEs are approximated by piecewise linear ODEs.
The advantage of piecewise linearization is that reachable sets of
linear ODEs can be computed far more efficiently than nonlinear ODEs.
However, the number of pieces required to ensure that the
linearization error is below a threshold increases exponentially in
the dimension of ODEs.  To our understanding, no effective solution
has yet been proposed to tackle the \emph{curse of dimensionality} in
piecewise linearization based flowpipe computation.  Some techniques
attempt to decompose the nonlinear system in higher dimensions into
subsystems with lower
dimensions~\cite{chen2018decomposition,chen2016decomposed}.  However,
the decomposition of nonlinear dynamics only scales effectively when
there exist small dimensional subsystems with loose coupling, but not
otherwise.

Alternatively, to avoid blowing up the number of pieces in piecewise
linearization, we can fix the number of pieces.  Then, we need to
optimize the way the reachable set is divided into subsets to minimize
the linearization error.  In this regard, prior work has been done by
Althoff. et. al.~\cite{althoff2008reachability} to optimally split
reachable sets for minimizing the linearization error.  They use a
single dimensional measure of reduction in linearization error while
splitting a reachable set.  However, the linearization error is a
multi-dimensional vector for which there is no single best optimum way
of splitting.  For different directions, there are different optimal
ways of splitting the reachable set.  Therefore, this splitting method
based on optimizing a single dimensional measure may not provide
desired accuracy along all coordinates.  This drawback is explained in
Section~\ref{sec:motivation} of our paper with an illustrative
example.

In this paper, we propose a more efficient way of minimizing the
multi-dimensional linearization error in flowpipe computation.  We use
different optimal divisions of a reachable set for minimizing the
linearization error along different directions, instead of a one way
of dividing the set.  Consequently, our reachable set is represented
as an intersection of the unions (IoU) of sets resulting from
different optimal ways of dividing the set, for different directions
of linearization error.  Then the linearization error is effectively
reduced along all directions in a given finite set of multiple
directions.  We develop an algorithm to propagate the intersection of
unions reachable set for flowpipe computation.  As a proof-of-concept,
we evaluate our algorithm on an illustrative example and two high
dimensional real-world examples. Our IoU flowpipe method showed a high
increase in the flowpipe accuracy compared to the earlier method of
division~~\cite{althoff2008reachability} and other state-of-the-art
methods~\cite{chen2012taylor,althoff2013reachability}.

In summary, we make the following contributions in this paper.
\begin{enumerate}
\item We introduce the use of intersection of unions, instead of
    just union, to minimize linearization error along multiple
    directions during flowpipe computation.  The use of intersection
    of union of sets is novel and has not been employed in previous
    research.
%
\item We develop a parallel algorithm to propagate IoU reachable sets
    over time.  The algorithm uses a variant of zonotope,
    called \emph{interval zonotope}, for computing intersection
    between zonotopes fastly and more accurately than simple zonotopes.
%
\item We demonstrate high increase in flowpipe accuracy on two high
    dimensional real world examples, compared to other state-of-the
    art methods.  Also, the computation time of our parallel
    algorithm is comparable or better than other approaches on
    multi-core machines.
\end{enumerate}
%
