The dynamics of many real world systems is modeled by ordinary
differential equations (ODEs).  These models typically contain
uncertainties in the parameters, inputs and initial conditions to
account for the unpredictable behavior of the system.  To ensure
reliable behavior of the systems, especially those which are safety
critical, these ODE models have to be verified to establish that they satisfy the
requirements for the working of the system.  Verifying properties of
the ODE models (eg. safety, stability) requires approximating the set
of trajectories (solutions) of the ODE in the presence of
uncertainties.  But exactly computing the set of trajectories of
nonlinear ODEs can be computationally intractable because of lack of
analytic solutions.  Alternatively, a lot of research has been done on
accurately overapproximating the set of trajectories by data
structures, referred to as \emph{flowpipes}.

Nevertheless, computing flowpipes with good accuracy is still a
formidable challenge.  The computationally complexity of most flowpipe
computation algorithms blow up beyond tractable limits when ensuring
good accuracy in high dimensions with large uncertainties in system
dynamics and initial conditions.  For instance, a commonly used method
in flowpipe computation is piecewise linearization, where the
nonlinear ODEs are approximated by piecewise linear ODEs.  The
advantage of piecewise linearization is that reachable sets of linear
ODEs can be computed far more efficiently than nonlinear ODEs.
However, the number of pieces required to ensure that the
linearization error is below a threshold increase exponentially in the
dimension of ODEs.  To our understanding, no effective solution has
yet been proposed to tackle the \emph{curse of dimensionality} in
piecewise linearization based flowpipe computation.  Some techniques
try to decompose the system in higher dimensions into subsystems with
lower dimensions to ensure
scalability~~\cite{chen2018decomposition,chen2016decomposed}.
However, the decomposition based methods only work well when there
there is loose coupling between some of the variables, but not for
systems with good coupling between the variables.

Alternatively, to avoid blowing up the number of pieces in piecewise
linearization, we can fix the number of pieces.  Then, we need to
optimize the way the reachable set is divided into subsets to minimize
the linearization error.  In this regard, prior work has been done by
Althoff. et. al.~\cite{althoff2008reachability} to optimally split
reachable sets for minimizing linearization error.  They use a single
dimensional measure of reduction in linearization error when a
reachable set is split into two and optimize this measure.  However,
the linearization error is a multi-dimensional vector for which there
is no single best optimum way of splitting.  For different directions,
there are different optimal ways of splitting the reachable set.
Therefore, this splitting method based on optimizing a single
dimensional measure may not give good accuracy along all coordinates.
This drawback is explained in Section~\ref{sec:motivation} with an illustrative
example.

In our paper, we propose a more efficient way of minimizing the
multi-dimensional linearization error in flowpipe computation.  We use
different optimal divisions of a reachable set for minimizing
linearization error along different directions, instead of a one way
of dividing the set.  Consequently, our reachable set is represented as
an intersection of the unions (IoU) of sets resulting from different
optimal ways of dividing the set for different directions of
linearization error.  Each union in this intersection corresponds to
minimizing the linearization error along one direction in a multiple
set of directions.  This way, we reduce the linearization error along
multiple directions.  As a proof-of-concept, we evaluate our algorithm
on an illustrative example and two high dimensional real-world
examples. Our IoU flowpipe method showed a high increase in the flowpipe accuracy
compared to the earlier method of division and other state-of-the-art
methods.

\subsection{Related work}
There are efficient algorithms to compute reachable sets of linear
uncertain
ODEs~\cite{girard2005reachability,FLD+11,girard2008efficient,bak2017simulation}.
The is because linear transformations of reachable sets can be
efficiently overapproximated.  In contrast, computing reachable sets
of nonlinear ODEs requires approximating nonlinear transformation of
reachable sets resulting in non-convex sets, which is a much harder
problem~\cite{dang2009image,monniaux2011generation}.  Taylor
models~\cite{chen2012taylor} and polynomial
zonotopes~\cite{althoff2013reachability,kochdumper2020sparse,kochdumper2020constrained}
have been developed to approximate the nonlinear transformation of
reachable sets.  But the complexity of Taylor model and polynomial
zonotope increases when nonlinear transformations are repeatedly
applied.  There are procedures to control this complexity, however,
the accuracy will reduce with increase in size of reachable sets.

Another approach used in numerous research works to flowpipe
computation of nonlinear ODEs is piecewise linear approximation of
nonlinear
systems~\cite{althoff2008reachability,li2020reachability,dang2010accurate,ramdani2009hybrid,han2006reachability}.
In this case, reachability analysis methods of affine hybrid systems
are used post piecewise linear approximation.  As discussed earlier,
this approach faces curse of dimensionality.  The number of pieces
required to restrict the linearization error increase exponentially in
the dimension.  Decomposition based
techniques~\cite{chen2018decomposition,chen2016decomposed} have been
proposed to address the curse of dimensionality, however, they they
work well under loose coupling between the state variables.  In
contrast, we show good accuracy of our IoU method on examples where
there is tight coupling between all state variables of ODE.
%
A idea similar to ours is a
\emph{projectahedron}~\cite{greenstreet1999reachability}, where the reachable
set is stored in terms of projections on different hyperplanes.  But,
storing projections of the reachable set can result in loosing the
correlation between variables in higher dimensions and consequently
affect flowpipe accuracy.  In contrast, our IoU method maintains a
good approximation of the correlation between variables by using a
variant of zonotope in its constituent sets.

The use of intersection of union of sets is novel and has not been
employed in previous research.  Each element of our IoU is a variant of
zonotope~\cite{girard2005reachability}, called \emph{interval
  zonotope}, introduced in this paper.  The interval zonotope is a
generalization of zonotope to allow better approximation of
intersection between zonotopes.  But the intersection of unions can
also use other representations like polytopes in its constituent sets.
We use this variant of zonotope because zonotopes are known to
approximate linear transformations very accurately.  Our IoU of
interval zonotopes is very different from zonotope
bundles~\cite{althoff2011zonotope}.  The latter is an intersection of
zonotopes but not intersection of union of zonotopes.

Our IoU flowpipe employs an optimization technique to construct the
IoU so as to minimize the linearization error along multiple
directions.  This is an improvement of the division method used in
Matthias et. al.~\cite{althoff2008reachability}.  The latter method
divides sets based on a single dimensional measure, which may not
effectively reduce linearization error along all directions.  Instead,
our IoU method uses multiple best divisions for minimizing
linearization along different directions, as discussed before.  This
improvement is explained latter in this paper more clearly with an
illustrative example.  Some methods approximate the unbounded time
union of reachable
sets~\cite{tiwari2008generating,prajna2006barrier,rodriguez2005generating}
instead of the reachable set at each individual time point. Such
approximation can verify boundedness within a safe set, but not other
temporal properties like reachability within a time interval.  Our
algorithm computes approximation at each time stamp in a time horizon
which can be helpful to verify temporal
properties.
