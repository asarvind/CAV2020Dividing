\begin{appendix}
\paragraph{Dynamics of quadrotor}
%
\begin{align*}
  \dot{p_n} & = u cos(\phi) cos(\theta) - v (cos(\phi)
  sin(\psi)-cos(\psi) sin(\phi) sin(\theta)) + \\ & w (sin(\phi) sin(\psi)+cos(\phi) cos(\psi) sin(\theta))\\
   \dot{p_e} & = v (cos(\phi) cos(\psi)+sin(\phi) sin(\psi)
  sin(\theta)) + u cos(\theta) sin(\psi) - \\ &w (cos(\psi) sin(\phi)-cos(\phi) sin(\psi) sin(\theta))  \\                          
   \dot{h} & = u sin(\theta) - w cos(\phi) cos(\theta)  - v cos(\theta) sin(\phi)\\
   \dot{u} & = r v - q w - g sin(\theta)\\
   \dot{v} & = p w - r u + g cos(\theta) sin(\phi)\\
   \dot{w} & = q u - p v + g cos(\phi) cos(\theta) - (m g - 10 (h - u_1) + 3 w)/m\\
   \dot{\phi} & = p + r cos(\phi) sin(\theta)/cos(\theta) + q sin(\phi) sin(\theta)/cos(\theta)\\
   \dot{\theta} & = q cos(\phi) - r sin(\phi)\\
   \dot{\psi} & = r cos(\phi)/cos(\theta) + q sin(\phi)/cos(\theta)\\
   \dot{p} & = (-(\phi - u_2) - p)/Jx + q r (Jy-Jz)/Jx\\
   \dot{q} & = (-(\theta - u_3) - q)/Jy + p r (Jx-Jz)/Jy\\
   \dot{r} & = 0/Jz + p q (Jx - Jy)/Jz\\
\end{align*}
%
\begin{proof}[Lemma~\ref{lem:inclin}]
Let us consider a trajectory $\trfn{x}:[0,T)\rightarrow\Omega$ of the
nonlinear system $H$.  Then there exists an input trajectory
$\trfn{u}:[0,T)\rightarrow U$ of $H$ such that~(\ref{eqn:dynamics}) is
true.  Let us define $\forall t\in[0,T)~\tr{v}{t} =
f\rb{\mymatrix{\tr{x}{t}^T & \tr{u}{t}^T}}^T - A\tr{x}{t} -
B\tr{u}{t}$.  Since $L$ is an overapproximation of $H$, we have
$\forall t\in[0,T)$, $\tr{v}{t}\in V$.  Then by using
~(\ref{eqn:dynamics}, we have the input trajectory
$\mymatrix{\trfn{u} \\ \trfn{v}}:[0,T)\rightarrow U\times V$ where
$\forall t\in[0,T), \epsilon\in\reals_{>0}$, there exists
$\delta\in\reals_{>0}$ such that $\forall
h\in[-\delta,\delta]\bigcap\, [0,T)$,
%
\begin{align*}
& \norm{ \frac{ \tr{x}{t+h} - \tr{x}{t} }{h} -
f_i\rb{ \mymatrix{ \tr{x}{t}^T & \tr{u}{t}^T }^T }}\\
& = \norm{ \frac{ \tr{x}{t+h} - \tr{x}{t} }{h} -
\rb{A\tr{x}{t} + B\tr{u}{t} + \tr{v}{t}} } < \epsilon.
\end{align*}
%
Therefore, $\tr{x}:[0,T)\rightarrow \reals^n$ is also a state
trajectory of the linear system $L$.  So, it follows that
$\reach{H}{t}{\init}\subseteq \reach{L}{t}{\init}$ for all
$\init\subseteq \Omega$.
\end{proof}
%
\begin{proof}[Lemma~\ref{lem:linreach}]
  It is known~(see Section 3.1 of~\cite{girard2005reachability}) that
the reachable set of the linear system at time $t$ is given by
%
\begin{equation*}
\begin{split}
& \left\{\exp\rb{At}x + \int_{\tau=0}^t\exp\rb{A\rb{t-\tau}}B\tr{u}{\tau}
d\tau\right.\\ &\left.
+ \int_{\tau=0}^t\exp\rb{A\rb{t-\tau}}\tr{v}{\tau}
d\tau\st{\forall \tau\in[0,t], \tr{u}{\tau}\in U,\tr{v}{\tau}\in
v}\right\}.
\end{split}
\end{equation*}
%
By using Taylor remainder theorem and interval arithmetic, we get for all
$t\in[t_1,t_2]~\exp\rb{At}\in
\dismat{A}{L}{[t_1,t_2]}$, $\forall \tau\in[0,t]~\exp\rb{A\rb{t-\tau}}B\in\ivmat{B}+\ivmat{B}\ivmat{A}[0,t],$
$\exp\rb{A\rb{t-\tau}}\in\ivmat{\identity{n}}+\ivmat{A}[0,t]$.  By
substituting these bounds in the above integrals, we get the result.
\end{proof}
%
\begin{proof}[Lemma~\ref{lem:bloat}]
We prove this by contradiction.  Let us denote $\Omega
= \ivflow{H}{t}{K}{\epsilon}{\bounds^0}$.  Let us assume that
$\Omega\nsupseteq \reach{H}{[0,t]}{\init}$.  So, $\Omega\neq\reals^n$
and $\Omega = \bounds^K$.  Then there exists a state trajectory
$\trfn{x}:[0,T)\rightarrow\reals^n$ and $\tau\in[0,t]$ such that $T>t$
and $\tr{x}{\tau}\notin \bounds^K$.  So, there exists $\tau_{min}$ in
$[0,T)$ such that $\tau_{min}
= \inf\set{t\in[0,T)\st{\tr{x}{\tau}\notin \bounds^K}}$.  By
using~(\ref{eqn:dynamics}), we can show there exists
$\delta\in[0,\infty)$ such that $\forall h\in[-\delta,\delta]$
$\tr{x}{\tau_{\min}+h}\in \ball{\tr{x}{\tau_{min}}}{\epsilon/2}$
($\epsilon$ is the positive real considered in the lemma statement).
Similarly, we can show there exists $r\in[0,\tau_{min})$ such that
$\tr{x}{r}\in \ball{\tr{x}{\tau_{min}}}{\epsilon/2}$.  Combining both
inequalities above, we get $\forall
h\in[-\delta,\delta],~\tr{x}{\tau_{\min}+h}\in \ball{\tr{x}{r}}{\epsilon}$.
As $r\in[0,\tau_{min})$, we get $\tr{x}{r}\in\bounds^K$.  So, for all
$h\in[-\delta,\delta)$, we have $\tr{x}{\tau_{\min}+h}\in \bounds^K +
[-\epsilon,\epsilon]^n$.  Also, by the definition of $\tau_{min}$, for
all $\tau\in[0,\tau_{min}-\delta/2)$, $\tr{x}{\tau}\in \bounds^K +
[-\epsilon,\epsilon]^n$. Combining last two inequalities, we get for
all $\tau\in[0,\tau_{min}+\delta)$ $\tr{x}{\tau}\in\bounds +
[-\epsilon,\epsilon]^n$.  Therefore,
$\trfn{x}|_{[0,\tau_{min}+\delta)}$ is a trajectory of
$\rb{f,U,\bounds^K+[-\epsilon,\epsilon]^n}$

By Lemma~\ref{lem:linearization}, $L_k$ is an overapproximation of
$\rb{f,U,\bounds^k+[-\epsilon,\epsilon]^n}$ .  As we showed that
$\trfn{x}|_{[0,\tau_{min}+\delta)}$ is a trajectory of
$\rb{f,U,\bounds^K+[-\epsilon,\epsilon]^n}$, using
Lemmas~\ref{lem:inclin} and~\ref{lem:linreach} and
Equation~\ref{eqn:iter}, we get that $\forall h\in [0,\delta)$,
$\tr{x}{\tau_{min}+h}\in \dismat{A}{L_K}{[0,t]}\bounds^0
+ \dismat{B}{L_K}{[0,t]}U
+ \dismat{C}{L_K}{[0,t]}V \subseteq \bounds^{k}$.  But by the
definition of $\tau_{min}$, $\exists h\in
[0,\delta),~\tr{x}{\tau_{min}+h}\notin \bounds^K$.  So, the assumption
that $\Omega\nsupseteq \reach{H}{[0,t]}{\init}$ is wrong.  Hence,
$\Omega\supseteq \reach{H}{[0,t]}{\init}$.
\end{proof}
%
\end{appendix}


