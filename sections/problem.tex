The problem of flowpipe computation is to find an overapproximation of
the reachable set at each time point in a horizon, called flowpipe,
such that essential reachability properties of the set, like safety
and convergence to a set, can be verified.  This usually requires
computing the linear and sometimes nonlinear projection of the
reachable set along any given direction.  In this paper, we compute
flowpipes with linear projection maps.
%
\begin{problem}
Let us consider an $n$-dimensional nonlinear system with $m$-inputs $H
= \rb{f,\Omega,U}$ where $U\in\intervals^m$ is bounded, i.e.,
$\sup\rb{U}\in\reals^n$.  We are given a bounded initial set
$X_0\in\intervals^n:~\sup{X_0}\in\reals^n$ and a time bound
$T\in\reals_{> 0}$.  For every $t\in[0,T]$, compute a set
$\tr{X}{t}\subseteq \reals^n$ and a function
$\flowproj{t}:\double^n\rightarrow \double$ such that
$\reach{H}{t}{X_0}\in\tr{X}{t}$ and $\forall
a\in\double^n~\sup\set{a^Tx\st{x\in\tr{X}{t}}}\leq
\flowproj{t}\rb{a}$.
\end{problem}
%
The set $\tr{X}{t}$ is called a 
\emph{flowpipe} at time $t$ and $\flowproj{t}$ is the flowpipe projection map.
For example in a $2$-dimensional system, a simple flowpipe is an
interval flowpipe where at any time $t$, we have
$\tr{X}{t}\in\intervals^2$ and $\flowproj{t}\rb{\tr{X}{t},a} =
\sup\rb{[a_1,a_1]\rb{\tr{X}{t}}_1 + [a_2,a_2]\rb{\tr{X}{t}}_2}$.

We would like the flowpipe approximation to be as close as possible to
the actual reachable set at any given time.  Greater accuracy of flowpipe results in
improved feasibility of verifying reachability properties of the
system.
