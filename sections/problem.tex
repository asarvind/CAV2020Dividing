An $n$-dimensional nonlinear dynamical system with $m$ inputs is
specified by a tuple $\rb{f,\inpset,\Omega}$, where $f =
\rb{f_1,\ldots,f_n}\in \fe{n+m}^n$ is an
$n$-tuple from the subset of partial functions $\fe{n+m}$, an
interval vector $\inpset\in\intervals^m$ is called the \emph{input
set}, and $\Omega\in\intervals^n$ is called the \emph{state space}.
For any $T\in\reals_{>0}$, a
function \mbox{$\trfn{x}:[0,T)\rightarrow \reals^n$} is called
a \emph{state trajectory} if there exists a \emph{piecewise
continuous} function \mbox{$\trfn{u}:[0,T)\rightarrow \inpset$},
called \emph{input trajectory}, such that all of the following are
true:
%
\begin{align*}
& \forall t\in[0,T), \tr{x}{t}\subseteq \Omega,\\
& \forall
t\in[0,T)\ \forall\epsilon\in\reals_{>0}, \exists\delta\in\reals_{>0}~\text{such
that}\\
& \ \ \ \  \forall h\in[-\delta,\delta]\bigcap[0,T],~\norm{ \frac{ \tr{x}{t+h}
- \tr{x}{t} }{h} - f_i\rb{ \mymatrix{ \tr{x}{t}\\ \tr{u}{t} }
}}< \epsilon.~\numberthis\label{eqn:dynamics}
\end{align*}
%
In the rest of the paper, $n$ is the dimension of a nonlinear system
and $m$ is the number of inputs, unless otherwise stated.  For a
trajectory $\trfn{x}:[0,T)\rightarrow \reals^n$ and $\tau<T$, we
define $\trfn{x}|_{[0,\tau]}
= \trfn{y}:[0,T)\rightarrow \reals^n: \forall t\in[0,\tau), \tr{y}{t}
= \tr{x}{t}$.   The \emph{reachable set} of the nonlinear system $H
=\rb{f,\inpset,\Omega}$ from an initial set $\init$ at a time point
$t$, denoted $\reach{H}{t}{\init}$, is a set containing all points
$\tr{x}{t}$, where $\trfn{x}$ is a state trajectory of $H$ and
$\tr{x}{0}\in\init$.  Similarly, $\reach{H}{[t_1,t_2]}{\init}
= \bigcup_{t\in[t_1,t_2]}\reach{H}{t}{\init}$.
%
The problem of flowpipe computation is to find an overapproximation of
the reachable set at each time point in a horizon, called flowpipe,
such that essential reachability properties of the set, like safety
and convergence to a set, can be verified.  This usually requires
computing the linear and sometimes nonlinear projection of the
reachable set along any given direction.  In this paper, we compute
flowpipes with linear projection maps.
%
\begin{problem}
Let us consider an $n$-dimensional nonlinear system with $m$-inputs $H
= \rb{f,U,\Omega}$, where $U\in\intervals^m$ is bounded, i.e.,
$\sup\rb{U}\in\reals^n$ \is{$\reals^m$?}.  We are given a bounded initial set
$X_0\in\intervals^n:~\sup(X_0)\in\reals^n$ and a time bound
$T\in\reals_{> 0}$.  For every $t\in[0,T]$, compute a set
$\tr{X}{t}\subseteq \reals^n$ and a function
$\flowproj{t}:\double^n\rightarrow \double$ such that
$\reach{H}{t}{X_0}\in\tr{X}{t}$ and $\forall
a\in\double^n~\sup\set{a^Tx\st{x\in\tr{X}{t}}}\leq
\flowproj{t}\rb{a}$.
\end{problem}
%
The set $\tr{X}{t}$ is called a 
\emph{flowpipe} at time $t$ and $\flowproj{t}$ is the flowpipe projection map.
%% For example, in a $2$-dimensional system, a simple flowpipe is an interval flowpipe where at any time $t$, we have
%% $\tr{X}{t}\in\intervals^2$ and $\flowproj{t}\rb{\tr{X}{t},a} =
%% \sup\rb{[a_1,a_1]\rb{\tr{X}{t}}_1 + [a_2,a_2]\rb{\tr{X}{t}}_2}$.
%
\begin{example}\label{eg:ill}
Let us denote the function expressions $\proj{1} = x$, $\proj{2} = y$,
$\proj{3} = \theta$ and $\proj{4} = u$.  A $3$-dimensional nonlinear
system with one input, state space $\reals^3$, and input set $U =
[-0.3,0.3]$ can be specified as $H = \rb{f,[-0.3,0.3],\reals^3}$, where
%
\begin{align*}
& f_1 = \rb{ u\rb{ x + \theta + 0.1y } - 1}x,
& f_2 = \rb{ u\rb{ y + \theta + 0.1x } - 1}y,\\
& f_3 = -\theta + 0.001\theta^2.
\end{align*}
%
Let us consider the zero input trajectory $\tr{u}{t} = 0~\forall
t\in[0,5)$.  In this case, two state trajectories are $\forall
t\in[0,5)$,
%
\begin{align*}
\tr{x}{t} = \mymatrix{ -0.1\exp\rb{-t}\\ 0.1\exp\rb{-t}\\ 0.1\exp\rb{-t} },~
& \tr{x}{t} = \mymatrix{ -0.2\exp\rb{-t}\\ -0.1\exp\rb{-t}\\ -0.1\exp\rb{-t} }
\end{align*}
%
which begin at $\tr{x}{0} = \mymatrix{-0.1,0.1,0.1}^T$ and $\tr{x}{0}
= \mymatrix{-0.2,-0.3,-0.1}^T$, respectively.
%
But there are also state trajectories without an analytic form.  The
objective of flowpipe computation is to compute bounds along different
directions on the reachable set $\reach{H}{t}{\init}$ at each time
point in $[0,T)$, which originate $\init$ (e.g., see Fig~\ref{fig:ill}).
%
We would like the flowpipe approximation to be as close as possible to
the actual reachable set at any given time.  Greater accuracy of flowpipe results in
improved feasibility of verifying reachability properties of the
system.
\end{example}
