 There are efficient algorithms to compute reachable sets of linear
 uncertain
 ODEs~\cite{girard2005reachability,FLD+11,girard2008efficient,bak2017simulation}.
 This is because linear transformations of reachable sets can be
 efficiently overapproximated.  In contrast, computing reachable sets
 of nonlinear ODEs requires approximating nonlinear transformation of
 reachable sets resulting in non-convex sets, which is a much harder
 problem~\cite{dang2009image,monniaux2011generation}.  Taylor
 models~\cite{chen2012taylor} and polynomial
 zonotopes~\cite{althoff2013reachability,kochdumper2020sparse,kochdumper2020constrained}
 have been developed to approximate the nonlinear transformation of
 reachable sets.  But the complexity of Taylor model and polynomial
 zonotope increases when nonlinear transformations are repeatedly
 applied.  There are procedures to control this complexity. However,
 the accuracy reduces with the increase in the size of the reachable
 sets.

Another approach used in numerous research works is piecewise linear
approximation of nonlinear
systems~\cite{althoff2008reachability,li2020reachability,dang2010accurate,ramdani2009hybrid,han2006reachability,bak2016scalable}.
In this case, reachability analysis methods of affine hybrid systems
are used post piecewise linear approximation.  As discussed earlier,
this approach faces curse of dimensionality.  For linear systems,
decomposition based techniques can be effectively used in high
dimensions~\cite{bogomolov2018reach}.  For nonlinear dynamics,
decomposition based
techniques~\cite{chen2018decomposition,chen2016decomposed} work well
under loose coupling between small dimensional subsystems.  In
contrast, our IoU method provides good accuracy for the systems where there
is tight coupling between all state variables of ODE.
%(see examples in Section~\ref{sec:eval}).  

An idea similar to ours is a
\emph{projectahedron}~\cite{greenstreet1999reachability}, where the reachable
set is stored in terms of projections on different hyperplanes.  But,
storing reachable set as projections on smaller dimensional
hyperplanes can result in loosing the correlation between variables in
higher dimensions and consequently affect flowpipe accuracy.  In
contrast, our IoU method maintains a good approximation of the
correlation between the variables by using a variant of zonotope in
its constituent sets.
%
Each element of our IoU is a variant of
zonotope~\cite{girard2005reachability}, called \emph{interval
  zonotope}, introduced in this paper.  The interval zonotope is a
generalization of zonotope to allow better approximation of
intersection between zonotopes.  But the intersection of unions can
also use other representations like polytopes in its constituent sets.
We use this variant of zonotope because zonotopes are known to
approximate linear transformations very accurately.  Our IoU of
interval zonotopes is very different from zonotope
bundles~\cite{althoff2011zonotope}.  The latter is an intersection of
zonotopes but not intersection of union of zonotopes.

Our IoU flowpipe employs an optimization technique to construct the
IoU so as to minimize the linearization error along multiple
directions.  This is an improvement of the division method used in
Matthias et. al.~\cite{althoff2008reachability}.  The latter method
divides the sets based on a single dimensional measure, which may not
effectively reduce linearization error along all directions.  Instead,
our IoU method uses multiple best divisions for minimizing
linearization along different directions, as discussed before.  This
improvement is explained latter in this paper more clearly with an
illustrative example.  Some methods approximate the unbounded time
union of reachable
sets~\cite{tiwari2008generating,prajna2006barrier,rodriguez2005generating}
instead of the reachable set at each individual time point. Such
approximation can verify boundedness within a safe set, but not other
temporal properties like reachability within a time interval.  Our
algorithm computes approximation at each time stamp in a time horizon
which can be helpful to verify temporal
properties.
