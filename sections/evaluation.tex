We evaluate our intersection of unions method on three examples, i.e.,
the $3$-dimensional system in the illustrative Example~\ref{eg:ill} in
Section~\ref{sec:system}, a $7$-dimensional nonlinear model of autonomous car
from Lavaei et. al.~\cite{lavaei2020formal}, $12$-dimensional
nonlinear model of quadrotor from ARCH 2020
workshop~\cite{geretti2020arch}.  For the illustrative example, we
compare the flowpipe accuracy of our intersection of unions (IoU)
method with that of only
\emph{union of interval zonotopes} based on the performance index in
Equation~\ref{eqn:pi} proposed by Althoff
et.al.~\cite{althoff2008reachability}.  
On the other two real world
examples, besides the performance index~(\ref{eqn:pi}), we compare our
method with polynomial
zonotopes~\cite{althoff2013reachability} in CORA
tool\footnote{\url{https://tumcps.github.io/CORA/}} and Taylor
models~\cite{chen2012taylor} in
Flowstar\footnote{https://flowstar.org}.

\is{Why don't we apply all these three methods on the three examples?}

For computation of symbolic derivatives, we use Sympy Python
software~\cite{10.7717/peerj-cs.103} and for interval arithmetic, we
use Boost C++ library~\cite{bronnimann2006design}.  We gain
significant speed by running parts of our algorithm in
parallel on mulicore machines.

\subsection{Illustrative example: IoU  vs union of interval zonotopes}
We consider the 3-dimensional nonlinear system in Example~\ref{eg:ill}
\is{Does it represent some real system, for example, a bicycle? It would have been nice then.}
to illustrate the difference in accuracy between our IoU flowpipe and
the union flowpipe based on the performance index
in~\cite{althoff2008reachability}(Equation~\ref{eqn:pi} above).  The
initial set has to be large enough to demonstrate the effect of linearization error, which is $X_0 = [-1,1]^3$.

\emph{Settings:}  We ran both the IoU and unions algorithm on a 1.4 GHz
laptop with 4 GB ram, 1600 MHz DDR3 with 4 virtual cpu cores.  We
consider 3 different number of divisions for comparision, i.e.,
$2^\eta:\eta = 2, 3,4$, or $4, 8$ and $16$ divisions,
respectively. The order of interval zonotope is $l = 200$ and the time
step size is $\delta = 0.01$ s.  Furthermore, $\epsilon = 10^{-12}$
and $K = 20$.  The normalization error $\rho$ for division in the union method is
the Taylor error before splitting $\taylor{f}{X_0}{U}$.

\emph{Results:}  The upper and lower bounds on flowpipes at uniformly spaced time
points for the $x$ and $y$ variables is
shown in Figure~\ref{fig:ill}.  The figure clearly demonstrates that our IoU
flowpipe is far more accurate than the union flowpipe for $x/y$
coordinates.  The bounds for $\theta$ are almost similar for both
methods and all types of divisions because of very less linearization
error $(\le 10^{-6})$.  The computation times for our IoU method are $6\si{\second}$, $12\si{\second}$ and $24\si{\second}$, respectively, for $4$, $8$ and $16$ divisions
per union.  The computation times will reduce significantly by using
more cpu cores.  We use more cpu cores for the higher dimensional
examples that we evaluate latter.
%
\begin{figure}
  \includegraphics[scale = 0.7]{illImages/Ub.png}
  %\includegraphics[scale=0.5]{illImages/ub.png}
  \caption{Flowpipe bounds at different time points for
    Example~\ref{eg:ill}}
  \label{fig:ill}
\end{figure}
%
\begin{figure}
\includegraphics[scale = 0.67]{autocarImages/ubToolSteering.png}
\includegraphics[scale = 0.41]{autocarImages/ubToolYaw.png}\hspace{-2.2em}
\includegraphics[scale = 0.41]{autocarImages/ubToolYawRate.png}
\includegraphics[scale = 0.41]{autocarImages/ubToolSlip.png}\hspace{-2.2em}
\includegraphics[scale = 0.41]{autocarImages/ubToolx2.png}
  \caption{Flowpipe bounds at different time points for
    car model}\label{fig:flowcar}
\end{figure}
%
%
\begin{table}
\caption{Computation times}\label{tab:comptimes}
\begin{tabular}{|l|c|c|c|c|c|}
\hline
Example & IoU  & IoU  & IoU  &
Polynomial & Taylor\\
& $\eta = 5$ & $\eta = 4$ & $\eta = 3$ & Zonotope & Model \\
\hline
& & & & & {\color{red}Incomplete}\\
Car & 252 s & 128 s & 108 s & 4068 s& {\color{red}[0,1.15s]:~216s}\\
\hline
& & & & &\\
Quadrotor & 169 s & 318 s & 614 s & 2382 s &
{\color{red}[0,4.13s]:~583 s} \\
\hline
\end{tabular}
\end{table}
%
\subsection{$7$-dimensional model of autonomous car}
A time discretized model of autonomous manoeuvre with $7$-dimensional
state space and $2$ inputs was presented in~\cite{lavaei2020formal}.
We adapted this model into a continuous time system and provided a
stabilizing feedback with one of the inputs.  Then we have the
following $7$-dimensional system having only one
uncontrolled input.
%
\begin{align*}
\dot{x_1}  = & x_4\cos\rb{x_5+x_7}\hspace{2em} \dot{x_2} =
x_4\sin\rb{x_5+x_7}\\
%
\dot{x_3}  = & -r(x_5+x_7+x_3) \hspace{2em} \dot{x_4} =
 u \hspace{2em} \dot{x_5} = x_6 & \\
 %
 \dot{x_6}  = & \frac{\mu
 m}{I_z(l_r+l_f)}(l_fC_{Sf}(gl_r-uh_{cg})x_3+
 (l_rC_{Sr}(gl_f+uh_{cg})\\& -l_fC_{Sf}(gl_r-uh_{cg}))x_7
 -(l_fl_fC_{Sf}(gl_r-uh_{cg}) + l_rl_rC_{Sr}(gl_f+uh_{cg}))\frac{x_6}{x_4})\\
%
\dot{x_7} 
= & \frac{\mu}{x_4(l_r+l_f)}(C_{Sf}(gl_r-uh_{cg})x_3+(C_{Sr}(gl_f+uh_{cg})-C_{Sf}(gl_r-uh_{cg}))x_7\\
&-(l_f*C_{Sf}(gl_r-uh_{cg}) + l_rC_{Sr}(gl_f+uh_{cg}))x_6/x_4)-x_6
\end{align*}
%
Above, $x = \rb{x_1,\ldots,x_7}^T$ represents the state vector and $u$
is the input, while rest are constant parameters.  The 2-D position of
car is $(x_1,x_2)$, steering angle is $x_3$, heading velocity is
$x_4$, yaw angle is $x_5$, yaw rate is $x_6$ and slip angle is $x_7$.
The parameter values are
%
$ g = 9.81, m = 1093.3, \mu = 1.0489, l_f = 1.156, l_r = 1.422, h_{cg}
  = 0.574, I_z = 1791.6, C_{Sf} = 20.89, C_{Sr} = 20.89, r = 4 $.  We
  consider the input set $u\in U = [-0.01,0.01]$ and the initial set
  $X_0 =
  [-1,1]\times[-0.5,0.5]\times[-0.5,0.5]\times[5,6]\times[-0.25,0.25]\times[-0.2,0.2]\times[-0.25,0.25]$.
  We compare our IoU method with the union
  method~(Equation~\ref{eqn:pi}) and also Taylor models in Flowstar
  with high expansion order and polynomial zonotopes in CORA with
  large number of generators and dependent generators.

\emph{Settings:}  The simulation time step is $\delta = 0.005$ s.  For
  the Flowstar Taylor model, the order of Taylor expansion is $9$.  For
  the CORA polynomial zonotope, generator order is $400$ and order of
  dependent generators is $2000$.  The zonotope order of our IoU
  method and the union method is $l=400$.  Also, $K = 20$ and
  $\epsilon = 10^{-12}$.  For the only union method, the normalization error
  error $\rho$ for division is the Taylor error before division
  $\taylor{f}{X_0}{u}$.  We ran our IoU method and the union method on
  an AWS c5a.16xlarge cluster with 64 virtual cpus.  For Flowstar, we
  used AWS t2.large instance.  We ran CORA in MATLAB 2019a on the
  computer specified in the previous illustrative example.

\emph{Results:}  The upper and lower bounds of flowpipe for different
  methods is plotted over times in $[0 s,5 s]$ in
  Figure~\ref{fig:flowcar}.  But Flowstar could not complete the Taylor
  model flowpipe simulation beyond 1.15 secs due to large
  time discretization error.  The bottom 3 lines in the figure correspond
  to our IoU flowpipe method, which show convergence while all other
  plots are diverging.  The figure clearly shows a high increase in
  accuracy by using intersection of unions compared to other methods.
  The simulation times are given in Table~\ref{tab:comptimes}.  The
  preprocessing time before simulation for IoU method is 18s.
  %
\subsection{$12$-dimensional model of quadrotor}
We consider the $12$-dimensional model of quadrotor with three inputs
presented in the ARCH 2020 friendly
competition~\cite{geretti2020arch}, whose dynamics is given in the
appendix.  It has a $12$-dimensional state vector $x
= \rb{p_n,p_e,h,u,v,w,\phi,\theta,\psi,p,q,r}$, where $h$ is the
height of the quadrotor and a $3$-dimensional input vector $u
= \rb{u_1,u_2,u_2}$.  We take a larger initial set that that
given in the competition~\cite{geretti2020arch} so that there is significant
linearization error in the flowpipe.  Our initial set is $X_0 =
[-0.8,0.8]^6\times[-0.5,0.5]^2\times[0,0]\times [-1,1]^2\times[0,0]$.
The input set is $\rb{u_1,u_2,u_3}\in
[-0.99,1.01]\times[-0.01,0.01]^2$.

\emph{Settings:}  The simulation time step is $\delta = 0.01$ s.  For
  Flowstar Taylor model, the order of Taylor expansion is $7$.  For
  the CORA polynomial zonotope, generator order is $200$ and order of
  dependent generators is $2000$.  The zonotope order of our IoU
  method and the union method is $l=200$.  Also, $K = 20$ and
  $\epsilon = 10^{-12}$.  For the only union method, we take the
  normalization error $\rho$ as the Taylor error before division
  $\taylor{f}{X_0}{u}$.  We ran our IoU method and the union method on
  an AWS c5a.16xlarge cluster with 64 virtual cpus.  For Flowstar, we
  used AWS t2.large instance.  We ran CORA in MATLAB 2019a on the
  computer specified in the previous illustrative example.  The
  simulation time horizon is $[0, 5 s]$.

\emph{Results:}  Our IoU method for each of the $8$, $16$ and $32$ divisions per
  eigenvector direction showed much higher accuracy than other
  methods.  The union method based on division using
  Equation~\ref{eqn:pi} from
  Althoff. et. al.~\cite{althoff2013reachability} resulted in very
  high linearization error and consequently infinite bounds
  $[-1,1]^{12}$ after $0.5 s$ simulation time, where the normalization
  error is taken to be $\rho = \taylor{f}{X_0}{u}$.  This also
  illustrates another drawback of the performance index proposed
  in~\cite{althoff2013reachability} that we do not know what is a good
  normalization error.  We plotted the flowpipe bounds for the height
  in Figure~\ref{fig:flowquadrotor}.  The figure clearly shows high
  increase in accuracy by using intersection of unions compared to
  other methods.  The computation times are given in Table~\ref{tab:comptimes}.
%
\begin{figure}
{\normalsize $\text{Note: Only union method resulted in infinite bounds (not plotted).}$}
%\vspace{-1.05em}
\includegraphics[scale = 0.65]{quadrotorImages/ubToolHeight.png}\vspace{-0.4em}
\includegraphics[scale = 0.6]{quadrotorImages/lbToolHeight.png}
\caption{Flowpipe bounds on height at different time points.}\label{fig:flowquadrotor}
\end{figure}
